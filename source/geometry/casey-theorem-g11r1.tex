% $date: 2017-11-08
% $timetable:
%   g11r1:
%     2017-11-08:
%       3:

\worksheet*{Теорема Кези}

% $authors:
% - Андрей Юрьевич Кушнир

%\definition
%\emph{Направленной окружностью или прямой} будем называть соответственно
%окружность или прямую, в~которой задано одно из~двух направлений обхода.
%Две направленные окружности или прямые касаются, если их направления в~точке
%касания совпадают.

\begin{problems}

\item\emph{(Техническая лемма для доказательства теоремы Кези)}
Направленные окружности $\omega_1$~и~$\omega_2$ радиусов $r_1$~и~$r_2$ касаются
направленной окружности~$\Omega$ радиуса~$R$ в~точках $A$~и~$B$ соответственно.
Докажите, что длина~$\delta_{12}$ общей направленной касательной
$\omega_1$~и~$\omega_2$ может быть выражена формулой
\[
    \delta_{12}
=
    \frac{AB}{R} \sqrt{(R \pm r_1)(R \pm r_2)}
\, . \]

\item\claim{Теорема Кези}
На~плоскости даны четыре направленные окружности
$\omega_1$, $\omega_2$, $\omega_3$,~$\omega_4$.
Обозначим длину общей направленной касательной для $\omega_i$~и~$\omega_j$
за~$\delta_{ij}$.
Тогда существует направленная окружность (или прямая), касающаяся их всех, если
и~только если выполнено соотношение
\[
    \pm \delta_{12}\delta_{34}
    \pm \delta_{13}\delta_{24}
    \pm \delta_{14}\delta_{23}
=
    0
\, . \]

\item
На~прямой в~указанном порядке отмечены точки $A$, $B$, $C$.
На~отрезке~$AC$ как на~диаметре построена окружность, на~которой отмечены
точки $P$~и~$Q$, так что $QB \perp AC$ и~$P$~и~$Q$ лежат в~разных
полуплоскостях относительно $AC$.
Докажите, что длина отрезка~$PQ$ равна сумме длин отрезков касательных из~$P$
к~окружностям, построенным на~отрезках $AB$ и~$BC$ как на~диаметрах.

\item\claim{Теорема Фейербаха}
Докажите, что окружность девяти точек треугольника касается вписанной
окружности.

\item
Докажите, что в~треугольнике расстояние от~точки Фейербаха до~середины одной
из~сторон равно сумме расстояний от~точки Фейербаха до~середин двух других.

\item\emph{Теорема Харта.}
У~криволинейного треугольника в~качестве сторон выступают дуги окружностей.
Рассмотрим его вписанную окружность и~три вневписанные.
Докажите, что существует окружность, которая касается и~вписанной, и~трех
вневписанных.

\item
На~вписанной окружности правильного треугольника $ABC$ отмечена случайная
точка~$P$, которую тут~же спроецировали на~стороны $BC$, $CA$, $AB$, получив
точки $P_{A}$, $P_{B}$, $P_{C}$ соответственно.
Окружность~$\omega_A$ касается $BC$ в~точке~$P_{A}$ и~касается меньшей
дуги~$BC$ описанной окружности треугольника $ABC$.
Окружности $\omega_B$ и~$\omega_C$ строятся аналогично.
Докажите, что сумма длин попарных общих внешних касательных окружностей
$\omega_A$, $\omega_B$, $\omega_C$ не~зависит от~выбора точки~$P$.

\item
Окружности $\omega_1$, $\omega_2$ и~$\omega_3$ касаются прямой~$\ell$
в~точках $A$, $B$ и~$C$ соответственно ($B$ лежит между $A$ и~$C$),
$\omega_2$ внешним образом касается двух других окружностей.
Пусть $X$ и~$Y$~--- точки пересечения $\omega_2$ со~второй общей внешней
касательной окружностей $\omega_1$ и~$\omega_3$.
Перпендикуляр, проведенный через точку~$B$ к~прямой~$\ell$, вторично пересекает
$\omega_2$ в~точке~$Z$.
Докажите, что окружность, построенная на~$AC$ как на~диаметре, касается
$ZX$ и~$ZY$.

\item
Пусть $ABC$~--- произвольный треугольник, а~$M$~--- точка внутри треугольника.
Проведем через~$M$ три чевианы, основания которых $A_{1}$, $B_{1}$, $C_{1}$.
Построим вне треугольника три окружности, касающиеся сторон треугольника
в~основаниях чевиан и~описанной окружности, и~четвертую, касающуюся этих трех
внешним образом.
Тогда эта окружность касается вписанной окружности треугольника внутренним
образом.

\end{problems}

