% $date: 2017-11-13
% $timetable:
%   g11r2:
%     2017-11-13:
%       3:

\worksheet*{Планиметрический разнобой}

% $authors:
% - Артемий Алексеевич Соколов

\begin{problems}

\item
Касательные к~описанной окружности остроугольного треугольника $ABC$,
восстановленные в~вершинах $B$~и~$C$, пересекаются в~точке~$X$.
Пусть $P$~и~$Q$~--- отражения точки~$X$ относительно прямых $AB$, $AC$.
Докажите, что прямая~$PQ$ проходит через середину $BC$.
%равные треугольники

\item
Вневписанные окружности треугольника $ABC$ касаются отрезков $AB$, $AC$
в~точках $C_1$, $B_1$ соответственно.
Докажите, что серединные перпендикуляры к~отрезкам $BB_{1}$, $CC_{1}$
пересекаются на~биссектрисе угла $BAC$.

\item
Пусть $C$~--- одна из~точек пересечения окружностей $\alpha$ и~$\beta$.
Касательная в~этой точке к~$\alpha$ пересекает $\beta$ в~точке~$B$,
а~касательная в~$C$ к~$\beta$ пересекает $\alpha$ в~точке~$A$, причем $A$ и~$B$
отличны от~$C$, и~угол $ACB$ тупой.
Прямая~$AB$ вторично пересекает $\alpha$ и~$\beta$ в~точках $N$ и~$M$
соответственно.
Докажите, что $2 MN < AB$.
% нер-во

\item
Через вершину~$B$ правильного треугольника $ABC$ проводится прямая~$\ell$.
Окружность~$\omega_{a}$ ($I_{a}$~--- центр) касается стороны~$BC$
в~точке~$A_1$, прямых $\ell$, $AC$.
Окружность~$\omega_{c}$ ($I_{c}$~--- центр) касается стороны~$BA$
в~точке~$C_1$, прямых $\ell$, $AC$.
Докажите, что ортоцентр треугольника $A_{1}BC_{1}$ лежит на~прямой
$I_{a}I_{c}$.

%\item
%Даны окружность, ее хорда~$AB$ и~точка~$W$~--- середина меньшей дуги~$AB$.
%На~большей дуге~$AB$ выбирается произвольная точка~$C$.
%Касательная к~окружности из~точки~$C$ пересекает касательные из~точек $A$ и~$B$
%в~точках $X$ и~$Y$ соответственно.
%Прямые $WX$ и~$WY$ пересекают прямую~$AB$ в~точках $N$ и~$M$ соответственно.
%Докажите, что длина отрезка~$NM$ не~зависит от~выбора точки~$C$.
%% радось

\item
Через вершины $B$ и~$C$ треугольника $ABC$ провели перпендикуляр прямой~$BC$
прямые $b$ и~$c$ соответственно.
Серединные перпендикуляры к~сторонам $AC$ и~$AB$ пересекают прямые $b$ и~$c$
в~точках $P$ и~$Q$ соответственно.
Докажите, что прямая~$PQ$ перпендикулярна медиане~$AM$ треугольника $ABC$.
% радось

\item
Пусть $\Gamma_1$ и~$\Gamma_2$~--- две непересекающиеся окружности.
$AB$~--- общая внешняя касательная окружностей,
$CD$~--- общая внутренняя касательная
($A$, $C$ принадлежат $\Gamma_1$, а~$B$, $D$ принадлежат $\Gamma_2$).
$AC$ и~$BD$ пересекаются в~точке~$E$.
$F$ принадлежит $\Gamma_1$, касательная к~$\Gamma_1$ в~$F$ пересекается
с~серединным перпендикуляром к~$EF$ в~точке~$M$.
$MG$~--- касательная к~$\Gamma_2$ из~точки~$M$.
Докажите, что $MF = MG$.
% радцентр

\item
На~стороне~$BC$ треугольника $ABC$ выбраны точки $P$ и~$Q$
($Q$ лежит между $P$ и~$C$) так, что $BP = QC$.
Описанная окружность треугольника $APQ$ пересекает стороны $AB$ и~$AC$
в~точках $E$ и~$F$ соответственно.
Пусть $T$~--- пересечение $EP$ и~$FQ$.
Прямые, проходящие через середину~$BC$ и~параллельные $AB$ и~$AC$, пересекают
$EP$ и~$FQ$ в~точках $X$, $Y$ соответственно.
Докажите, что описанные окружности $TXY$ и~$APQ$ касаются.
% хард

\end{problems}

