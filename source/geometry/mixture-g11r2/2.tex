% $date: 2017-11-15
% $timetable:
%   g11r2:
%     2017-11-15:
%       1:

\worksheet*{Планиметрический разнобой~--- 2}

% $authors:
% - Артемий Алексеевич Соколов

\begin{problems}

% Утром займусь плотно, сейчас вырубает.
% Планирую добавить несколько задачек с в том числе проективными решениями.

\item
Через вершины треугольника $ABC$ проведены 6~чевиан, лежащих внутри
треугольника и~образующих с~противоположными сторонами 6 равных острых углов.
Докажите, что среди 12~точек пересечения этих отрезков найдутся 6, лежащих
на~одной окружности.

\item
Дана прямая~$\ell$ и~точки $A$, $B$ с~одной стороны от~нее.
Постройте циркулем и~линейкой такую точку~$P$ на~прямой~$\ell$, что
$\angle(\ell, PA) = 2 \angle(PB, \ell)$.
% Может и не в тему, но мне нравится эта задача.
% Да и какая тема, разнобой же

\item
На~окружности~$\omega$ с~диаметром~$AB$ отмечена точка~$C$.
Касательные к~$\omega$ в~точках $B$~и~$C$ пересекаются в~точке~$P$.
Пусть $M$~--- середина меньшей дуги~$AC$, $N$~--- середина отрезка~$BC$.
Прямая~$MN$ пересекает~$\omega$ в~точках $M$~и~$Q$.
Докажите, что $\angle PQM = 90^{\circ}$.

\item
В~треугольнике $ABC$ проведена биссектриса~$AD$.
Вписанная окружность треугольника $ABD$ касается сторон $AB$, $AD$, $BD$
в~точках $P$, $U$, $M$.
Вневписанная окружность треугольника $ACD$ касается отрезка~$CD$ в~точке~$N$
и~прямых $AC$, $AD$ в~точках $Q$~и~$W$.
Докажите, что $W$ лежит на~прямой~$PM$, а~$U$ лежит на~прямой~$QN$.

\item
Даны окружность, ее хорда~$AB$ и~точка~$W$~--- середина меньшей дуги~$AB$.
На~большей дуге~$AB$ выбирается произвольная точка~$C$.
Касательная к~окружности из~точки~$C$ пересекает касательные из~точек $A$ и~$B$
в~точках $X$ и~$Y$ соответственно.
Прямые $WX$ и~$WY$ пересекают прямую~$AB$ в~точках $N$ и~$M$ соответственно.
Докажите, что длина отрезка~$NM$ не~зависит от~выбора точки~$C$.

\item
Докажите, что основания перпендикуляров, опущенных из~вершин $A$~и~$C$
описанного четырехугольника $ABCD$ на~биссектрисы углов $ABC$~и~$ADC$, лежат
на~одной окружности.

\item
В~параллелограмме $ABCD$ проведена высота~$BH$ к~стороне~$AD$.
Две окружности, проходящие через точки $C$~и~$D$, касаются прямой~$BH$ в~точках
$X$~и~$Y$.
Пусть $M$~--- середина $AB$.
Докажите, что $MD$~--- биссектриса угла $XMY$.

\end{problems}

