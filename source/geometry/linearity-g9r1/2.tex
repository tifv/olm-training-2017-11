% $date: 2017-11-11
% $timetable:
%   g9r1:
%     2017-11-11:
%       3:

\worksheet*{Линейность. Теорема о~трёх положениях}

% $authors:
% - Андрей Юрьевич Кушнир

\begin{problems}

\item\emph{Теорема о~трёх положениях.}
Докажите, что если три линейно движущиеся точки в~три различных момента времени
оказались на~одной прямой, то~они всегда на~одной прямой.
\begin{em}
Наводящий вопрос: какое условие коллинеарности векторов с~координатами
$(x_1, y_1)$ и~$(x_2, y_2)$?
\end{em}

\item\emph{Прямая Гаусса.}
На~плоскости проведено четыре прямые общего положения.
Докажите, что середины трех отрезков, соединяющих точку пересечения двух прямых
с~точкой пересечения двух оставшихся (и~так для трех разбиений прямых на~пары),
лежат на~одной прямой.
% Тащить прямую — довольно необычно.

\item
На~основании~$BC$ равнобедренного треугольника $ABC$ ($AB = AC$) отмечена
точка~$X$.
На~прямой, проходящей через~$C$ параллельно $AB$, выбрана точка~$Y$ так, что
$XY \perp AC$.
Пусть $Z$~--- центр описанной окружности треугольника $ABX$, а~$D$~---
середина $BC$.
Докажите, что $\angle YDZ = 90^{\circ}$.

\item
Докажите, что середины трех отрезков, соединяющих проекции произвольной точки
плоскости на~пары противоположных сторон или диагоналей вписанного в~окружность
четырехугольника, лежат на~одной прямой.

%\item
%Точка~$M$~--- середина стороны~$BC$ треугольника $ABC$.
%На~отрезках $BM$ и~$CM$ выбраны точки $P$~и~$Q$ соответственно таким образом,
%что $PQ = BC / 2$.
%Окружность, описанная около треугольника $ABQ$, пересекает сторону~$AC$
%в~точке $X \neq A$, а~окружность, описанная около треугольника $ACP$,
%пересекает сторону~$AB$ в~точке $Y \neq A$.
%Докажите, что четырехугольник $AXMY$~--- вписанный.

\item
На~сторонах $AB$, $AC$ треугольника $ABC$ отмечены точки $B_1$, $C_1$
соответственно.
Докажите, что прямая, соединяющая ортоцентры
треугольников $ABC$ и~$AB_{1}C_{1}$, перпендикулярна прямой, соединяющей точки
пересечения медиан треугольников $ABC_{1}$ и~$AB_{1}C$.

\item
На~высотах остроугольного треугольника $ABC$ из~вершин $A$, $B$, $C$ отметили
точки $A_1$, $B_1$, $C_1$ соответственно, а~$H$~--- ортоцентр.
Докажите, что $A_1$, $B_1$, $C_1$ и~$H$ лежат на~одной окружности тогда
и~только тогда сумма площадей треугольников $A_{1}BC$, $B_{1}CA$ и~$C_{1}AB$
равна площади треугольника $ABC$.

\item
Пусть $O$~--- центр описанной окружности остроугольного треугольника $ABC$,
в~котором $\angle A < 60^{\circ}$.
Обозначим $B'$, $C'$, $O'$ образы точек $B$, $C$, $O$ при отражении
относительно прямых $AC$, $AB$, $BC$ соответственно.
Докажите, что прямая~$AO'$ содержит диаметр описанной окружности треугольника
$AB'C'$.

\item
Пусть $O$~и~$H$~--- соответственно центр описанной окружности и~ортоцентр
остроугольного неравнобедренного треугольника $ABC$, а~$AP$ и~$AM$~---
соответственно высота и~медиана этого треугольника.
Пусть $D$~--- проекция точки~$A$ на~прямую~$OH$.
Окружность~$\omega$ с~центром~$S$ проходит через точки $A$~и~$D$ и~пересекает
вторично отрезки $AB$ и~$AC$ в~точках $X$~и~$Y$ соответственно.
Докажите, что центр описанной окружности треугольника $XSY$ равноудален
от~$P$~и~$M$.
% 10.6 всероса где еще Золотов был (давал весной 2017 в~первом разнобою 10-2).

\end{problems}

