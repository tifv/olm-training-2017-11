% $date: 2017-11-11
% $timetable:
%   g9r1:
%     2017-11-11:
%       2:

\worksheet*{Линейность. Вводные задачи}

% $authors:
% - Андрей Юрьевич Кушнир

% То есть по возможности без теоремы о трех положениях либо с попыткой
% придумать ее естественно в процессе решения.

\definition
Пусть для каждого $t \in \mathbb{R}$ определена точка~$A(t)$
(вектор~$\vec{a}(t)$ или прямая~$\ell(t)$).
Будем говорить, что $A(t)$ ($\vec{a}(t)$ или $\ell(t)$)
\emph{линейно зависит от~$t$} или просто \emph{линейно движется}, если
существует такой вектор~$\vec{v}$, что
$A(t) = A(0) + t \cdot \vec v$
($\vec{a}(t) = \vec{a}(0) + t \cdot \vec{v}$ или
$\ell(t) = \ell(0) + t \cdot \vec{v}$).
Под обозначением <<$\text{объект} + \vec{w}$>> имеется ввиду \text{объект},
перенесенный на~вектор $\vec w$ (или просто сумма векторов).

\begin{problems}

% Начну с разбора вот этого тупого примера:

%\item
%На~стороне~$CD$ квадрата $ABCD$ отмечена точка~$A_1$.
%На~отрезке~$A_{1}D$ как на~стороне вовне построен квадрат $A_{1}B_{1}C_{1}D$.
%Прямые $AC$ и~$A_{1}C_{1}$ пересекаются в~точке~$X$, про которую нужно доказать,
%что она лежит на~прямой~$BB_{1}$.

\item
Докажите или опровергните:
\\
\subproblem
Вектор, соединяющий две линейно движущиеся точки, меняется линейно.
\\
\subproblem
Середина отрезка, соединяющего две линейно движущиеся точки, движется линейно.
\\
\subproblem
Прямая через две линейно движущиеся точки движется линейно.
\\
\subproblem
Точка пересечения линейно движущихся прямых движется линейно.
\\
\subproblem
Пучок параллельных прямых $\ell(t)$, на~каждой из~которых удалось отметить
линейно зависящую от~$t$ точку $A(t)$, также линейно зависит от~$t$.

\item
Вписанная в~треугольник $ABC$ окружность касается его сторон $AB$, $AC$
в~точках $C_1$, $B_1$ соответственно.
На~отрезках $BC_{1}$, $AB_{1}$ отмечены точки $P$ и~$Q$ соответственно, что
$PC_{1} = QB_{1}$.
Докажите, что середина отрезка~$PQ$ лежит на~прямой~$B_{1}C_{1}$.

%\item
%Диагонали выпуклого четырехугольника перпендикулярны.
%Докажите, что перпендикуляры из~середин двух соседних сторон к~противоположным
%сторонам пересекаются на~диагонали.
%% Проще по-человечески гомотетией.

\item
Пусть $M$~--- середина стороны~$BC$ треугольника $ABC$.
На~его сторонах $AB$, $AC$ отмечены точки $C_1$~и~$B_1$ соответственно, причем
$\angle AB_{1}M = \angle AC_{1}M$.
Докажите что перпендикуляры, восстановленных из~точек $B_1$, $C_1$, $M$
к~сторонам треугольника, на~которых они лежат, пересекаются в~одной точке.

\item
На~стороне~$BC$ равностороннего треугольника $ABC$ с~центром~$I$ отмечена
точка~$X$.
Из~точки~$X$ опустили перпендикуляры $XP$ и~$XQ$ на~стороны $AB$ и~$AC$
соответственно.
Докажите, что прямая~$XI$ делит отрезок~$PQ$ пополам.

\item
Некоторая окружность с~центром в~точке~$I$ пересечения биссектрис
треугольника $ABC$ пересекает стороны треугольника
в~точках $A_{B}$, $A_{C}$, $B_{C}$, $B_{A}$, $C_{A}$, $C_{B}$
(обозначения: точка~$X_{Y}$ лежит на~стороне напротив вершины~$X$ и~ближе
к~точке~$Y$).
Докажите, что прямая, соединяющая $A$ с~точкой пересечения
$C_{B}A_{B}$ и~$B_{C}A_{C}$, делит отрезок~$A_{B}A_{C}$ пополам.
% Хитрое положение (время нужно отмотать назад).

%\item
%Вневписанные окружности касаются сторон $AB$, $AC$ треугольника $ABC$
%в~точках $C_1$, $B_1$ соответственно.
%Докажите, что прямая, соединяющая середины $BC$ и~$B_{1}C_{1}$, параллельна
%биссектрисе угла~$A$.
%% Надо пошевелить жесткое условие задачи.

\item
Окружность~$\omega$ с~центром~$O$ вписана в~угол $BAC$ и~касается его сторон
в~точках $B$~и~$C$.
Внутри угла $BAC$ выбрана точка~$Q$.
На~отрезке~$AQ$ нашлась такая точка~$P$, что $AQ \perp OP$.
Прямая~$OP$ пересекает окружности $\omega_1$~и~$\omega_2$, описанные около
треугольников $BPQ$ и~$CPQ$, вторично в~точках $M$~и~$N$.
Докажите, что $OM = ON$.

\item
На~стороне~$BC$ параллелограмма $ABCD$ ($\angle A < 90^{\circ}$) отмечена
точка~$T$ так, что треугольник $ATD$~--- остроугольный.
Пусть $O_1$, $O_2$ и~$O_3$~--- центры описанных окружностей
треугольников $ABT$, $DAT$ и~$CDT$ соответственно.
Докажите, что точка пересечения высот треугольника $O_{1}O_{2}O_{3}$ лежит
на~прямой~$AD$.
% Хитрое движение.

%\item
%К~двум не~пересекающимся окружностям проведены внешняя $PM$ и~внутренняя $QN$
%касательные.
%Точки $P$, $Q$ лежат на~одной из~окружностей;
%$M$, $N$~--- на~другой.
%Докажите, что точка пересечения $PQ$ и~$MN$ лежит на~линии центров.

\end{problems}

