% $date: 2017-11-14
% $timetable:
%   g11r1:
%     2017-11-14:
%       1:

\worksheet*{Геометрия окружностей на~сфере}

% $authors:
% - Андрей Юрьевич Кушнир

\begingroup
    \def\spherex#1{S(#1)}

Для каждой точки~$P$, находящейся вне сферы~$\omega$, обозначим через
$\spherex{P}$ окружность, содержащую все основания касательных из~$P$
к~$\omega$.

\begin{problems}

\item
Докажите, что
\\
\subproblem
если прямая~$AB$ касается сферы, то~$\spherex{A}$ и~$\spherex{B}$ касаются.
\\
\subproblem
если плоскость $(ABC)$ касается сферы,
то~$\spherex{A}$, $\spherex{B}$, $\spherex{C}$ пересекаются в~одной точке.

\end{problems}

\claim{Мысль}
Таким образом, конфигурация точек в~пространстве переносится на~конфигурацию
окружностей на~сфере.
Далее, применив стереографическую проекцию в~какой-нибудь точке сферы
(т.\,е. инверсию), мы получим планиметрическую задачу.

\begin{problems}

\item
Около сферы описан пространственный четырехугольник.
Докажите, что четыре точки касания лежат в~одной плоскости.

\item
В~четырехугольную пирамиду $SABCD$ вписана сфера.
На~ребрах $SA$, $SB$, $SC$, $SD$ отмечены точки $A_1$, $B_1$, $C_1$, $D_1$
соответственно.
Докажите, что вторые касательные плоскости к~сфере, проведенные через ребра
$A_{1}B_{1}$, $B_{1}C_{1}$, $C_{1}D_{1}$, $D_{1}A_{1}$, пересекаются в~одной
точке.

\item
На~ребрах произвольного тетраэдра выбрано по~точке.
Через каждую тройку точек, лежащих на~ребрах с~общей вершиной, проведена
плоскость.
Докажите, что если три из~четырех проведенных плоскостей касаются вписанного
в~тетраэдр шара, то~и~четвертая плоскость также его касается.

%\item
%Семь из~восьми вершин многогранника, комбинаторно эквивалентного кубу, лежат
%на~сфере.
%Докажите, что восьмая вершина также лежит на~этой сфере.

\item
Сфера касается всех ребер тетраэдра $ABCD$, кроме, быть может, ребра~$BD$.
Оказалось, что точки касания сферы с~ребрами $AB$, $BC$, $CD$, $DA$ в~указанном
порядке являются вершинами квадрата.
Докажите, что сфера все-таки касается ребра~$BD$.

%\item
%В~тетраэдр $SABC$ вписана сфера.
%На~ребрах $SA$, $SB$, $SC$ отмечены пары точек
%$A_1$ и~$A_2$, $B_1$ и~$B_2$, $C_1$ и~$C_2$ соответственно.
%Оказалось, что плоскости
%$AB_{1}C_{1}$, $BC_{1}A_{1}$, $CA_{1}B_{2}$, $AB_{2}C_{2}$, $BC_{2}A_{2}$
%касаются сферы.
%Докажите, что $CA_{2}B_{1}$ тоже касается сферы.

\end{problems}

Для решения задач могут понадобиться более интересные факты.
Изучите следующую часть листика, прочитайте все формулировки.
Можно доказывать свойства по~мере необходимости для решения задач.

\begin{problems}

\item
К~сфере проведены две скрещивающиеся касательные $\ell_1$ и~$\ell_2$.
На~них выбираются случайные точки $A_1$ и~$A_2$ соответственно таким образом,
что прямая~$A_{1}A_{2}$ касается сферы в~точке~$X$.
Докажите, что все точки~$X$ лежат на~некоторой окружности.

\item
Дана сфера и~две точки вне нее.
Докажите, что геометрическое место точек пересечения касательных к~сфере,
проходящих через эти точки, можно засунуть в~объединение двух плоскостей.
% Здесь понадобится понятие связки окружностей, сложно.

\item
Вписанная в~четырехугольную пирамиду $SABCD$ сфера~$\Xi$ касается
граней $SBC$ и~$SDC$ в~точках $X$ и~$Y$ соответственно.
Пары лучей $AB$ и~$DC$, $AD$ и~$BC$ пересекаются в~точках $P$ и~$Q$
соответственно.
Докажите, что прямые $PX$ и~$QY$ пересекаются тогда и~только тогда, когда $\Xi$
касается отрезка~$AC$.

\item
Дан выпуклый четырехугольник $ABCD$.
Внутри него отмечаются всевозможные пары точек $P$ и~$Q$ таких, что
$\angle APB = \angle AQB = \angle CPD  = \angle CQD$.
Докажите, что все прямые~$PQ$ проходят через одну точку или параллельны.

\end{problems}

Ниже можно найти интересные факты об~изучаемом соответствии $S_{(X)}$ между
точками проективного пространства вне сферы и~окружностями на~сфере (а~после
инверсии и~на~плоскости).
\par
\emph{Не~везде даны точные формулировки свойств, кое-где предлагается дать
определение или ответить на~вопрос.
Это скорее листик-исследование.}

\begin{enumerate}

\item
Определите полярное соответствие относительно сферы
(полярную плоскость точки, поляру прямой).
Убедитесь в~полярной двойственности: если точка~$A$ лежит на~полярной плоскости
точки~$B$, то~$B$ лежит на~полярной плоскости $A$.
Точки $A$ и~$B$ в~этом случае называются \emph{сопряжёнными}.

\item
Образом прямой пространства при соответствии $\spherex{X}$ служит
\emph{пучок окружностей.}
Кстати, пучок окружностей~--- максимальное по~включению множество окружностей
с~общей радикальной осью.
Пучки бывают эллиптическими, гиперболическими и~параболическими
(посмотрите картинки в~интернете).
Как по~прямой определить, какой из~нее пучок получится?
Какие еще множества окружностей или прямых естественно считать пучками
(т.\,е. в~каких еще ситуациях они образы некоторых прямых пространства)?

\item
Какие пары окружностей $\spherex{A}$, $\spherex{B}$ соответствуют парам
сопряженных точек $A$ и~$B$?
Докажите, что для каждого пучка окружностей множество окружностей,
ортогональных им всем, образует пучок, причем противоположного типа.

\item
Докажите, что три окружности лежат в~одном пучке тогда и~только тогда, когда их
уравнения (в~декартовых координатах) линейно независимы.
Алгебраический взгляд на~пучки окружностей иногда бывает продуктивен.
К~примеру, можно доопределить наше соответствие $S_{\cdot}$ на~точках внутри
сферы: сопоставим им уравнения окружностей с~мнимым радиусом.

\item
Образом плоскости при соответствии $\spherex{X}$ является
\emph{связка} окружностей.
Дайте эквивалентные определения связки в~терминах геометрии плоскости и~в~виде
подпространства пространства уравнений.

\item
Докажите, что круговые преобразования плоскости (порожденные инверсией
и~подобиями) сохраняют пучки и~связки окружностей и~индуцируют с~помошью
$\spherex{X}^{-1}$ проективное преобразования в~пространстве, сохраняющее
сферу.
Убедитесь, что так может быть получено любое преобразование, сохраняющее сферу.

\item
Аналогичное соответствие $T(X)$ определим по-другому:
точке с~однородными координатами $X = (A, B, C ,D)$ сопоставим
окружность $A(x^2 + y^2) + Bx + Cy + Z = 0$.
Докажите, что $T( \spherex{X}^{-1} )$~--- круговое преобразование.
\par
\emph{Т.\,е., можно было~бы определять наше соответствие вот так по-простому
алгебраически, и~все свойства~бы сохранились.}

\end{enumerate}

\endgroup % \sphereX

