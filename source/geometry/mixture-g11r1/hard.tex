% $date: 2017-11-09
% $timetable:
%   g11r1:
%     2017-11-09:
%       3:

\worksheet*{Сложный геометрический разнобой}

% $authors:
% - Андрей Юрьевич Кушнир

\begin{problems}

\item
Хорды $AC$ и~$BD$ окружности $\Omega$ пересекаются в~точке~$X$.
Докажите, что радикальная ось окружностей, вписанных в~криволинейные
треугольники $AXB$ и~$CXD$, проходит через середины дуг $BC$ и~$DA$.

\item
Дан выпуклый шестиугольник $ABCDEF$.
Докажите, что отрезки, соединяющие середины  противоположных сторон,
пересекаются в~одной точке тогда и~только тогда, когда треугольники $ACE$
и~$BDF$ равновелики.

\item
Дан выпуклый четырехугольник $ABCD$, у~которого $BA \neq BC$.
Обозначим окружности, вписанные в~треугольники $ABC$ и~$ADC$, через
$\omega_1$ и~$\omega_2$.
Предположим, что существует окружность~$\omega$, касающаяся продолжения
отрезка~$BA$ за~точку~$A$, продолжения $BC$ за~точку~$C$,
продолжений $CD$ и~$AD$ за~точку~$D$.
Докажите, что общие внешние касательные к~$\omega_1$ и~$\omega_2$ пересекаются
на~$\omega$.

\item
A triangle $ABC$ is inscribed in a circle~$\omega$.
A variable line~$\ell$ chosen parallel to $BC$ meets segments $AB$, $AC$ at
points $D$, $E$ respectively, and meets $\omega$ at points $K$, $L$
(where $D$ lies between $K$ and $E$).
Circle~$\gamma_1$ is tangent to the segments $KD$ and $BD$ and also tangent
to $\omega$, while circle~$\gamma_2$ is tangent to the segments $LE$ and $CE$
and also tangent to $\omega$.
Determine the locus, as $\ell$ varies, of the meeting point of the common inner
tangents to $\gamma_1$ and $\gamma_2$.

\item
Докажите, что радикальный центр трех полувписанных окружностей
треугольника $ABC$ лежит на~прямой, соединяющей центры вписанной и~описанной
окружностей.

\item
Остроугольный треугольник $ABC$ вписан в~окружность~$\Omega$.
Пусть $\ell$~--- некоторая касательная к~$\Gamma$, и~пусть
$\ell_{a}$, $\ell_{b}$, $\ell_{c}$~--- отражения прямой~$\ell$
относительно $BC$, $CA$, $AB$.
Докажите, что окружность, описанная относительно треугольника, образованного
прямыми $\ell_{a}$, $\ell_{b}$, $\ell_{c}$, касается $\Omega$.

\item
Диагонали $AC$ и~$BD$ вписанного в~окружность четырехугольника $ABCD$
пересекаются в~точке~$X$.
Докажите, что прямые Эйлера треугольников $ABX$, $BCX$, $CDX$, $DAX$
пересекаются в~одной точке или попарно параллельны.

\end{problems}

