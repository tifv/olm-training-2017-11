% $date: 2017-11-09
% $timetable:
%   g11r1:
%     2017-11-09:
%       3:

\worksheet*{Не~очень сложный геометрический разнобой}

% $authors:
% - Андрей Юрьевич Кушнир

\begin{problems}

\item
Касательные к~описанной окружности остроугольного треугольника $ABC$,
восстановленные в~вершинах $B$~и~$C$, пересекаются в~точке~$X$.
Пусть $P$~и~$Q$~--- отражения точки~$X$ относительно прямых $AB$, $AC$.
Докажите, что прямая~$PQ$ проходит через середину $BC$.

\item
Высоты $BB_{1}$ и~$CC_{1}$ остроугольного неравнобедренного треугольника $ABC$
пересекаются в~точке~$H$.
Отрезки $AH$ и~$B_{1}C_{1}$ пересекаются в~точке~$P$.
Прямая~$AO$ пересекает отрезок~$BC$ в~точке~$Q$
($O$~--- центр описанной окружности треугольника $ABC$),
а~$M$~--- середина~$BC$.
Докажите, что $MH \parallel PQ$.

\item
Вневписанные окружности треугольника $ABC$ касаются отрезков $AB$, $AC$
в~точках $C_1$, $B_1$ соответственно.
Докажите, что серединные перпендикуляры к~отрезкам $BB_{1}$, $CC_{1}$
пересекаются на~биссектрисе угла $BAC$.

\item
Докажите, что основания перпендикуляров, опущенных из~вершин $A$~и~$C$
описанного четырехугольника $ABCD$ на~биссектрисы углов $ABC$~и~$ADC$, лежат
на~одной окружности.

\item
В~треугольнике $ABC$ ($AB < AC$) точка~$I$~--- центр вписанной окружности,
$M$~--- середина $BC$, $N$~--- середина дуги $BAC$ описанной окружности
треугольника.
Докажите, что $\angle IMB = \angle INA$.

\item
В~неравнобедренном треугольнике $ABC$ биссектрисы углов $B$~и~$C$ пересекаются
в~точке~$I$, пересекают стороны $AC$, $AB$ в~точках $B_1$, $C_1$, а~также
пересекают <<меньшие>> дуги $AC$, $AB$ описанной окружности
в~точках $B_0$, $C_0$ соответственно.
Прямые $B_{0}C_{0}$ и~$B_{1}C_{1}$ пересекаются в~точке~$X$.
Докажите, что $XI \parallel BC$.

\item
В~остроугольный неравнобедренный треугольник $ABC$ вписан произвольный
соответственно подобный ему треугольник $A_{1}B_{1}C_{1}$
($A_1$ лежит на~$BC$, $B_1$ лежит на~$CA$, $C_1$ лежит на~$AB$);
$H$~--- ортоцентр треугольника $AB_{1}C_{1}$.
Докажите, что длина отрезка $A_{1}H$ не~зависит от~выбора
треугольника $A_{1}B_{1}C_{1}$.

%\item
%Дан выпуклый четырехугольник $ABCD$.
%Внешние биссектрисы пар углов $A$ и~$B$, $B$ и~$C$, $C$ и~$D$, $D$ и~$A$
%пересекаются в~точках $P$, $Q$, $R$, $S$ соответственно.
%Оказалось, что описанные окружности треугольников $ABP$, $CDR$ касаются
%(внешним образом).
%Докажите, что описанные окружности треугольников $BCQ$, $DAS$ тоже касаются.

\item
Хорды $AC$ и~$BD$ окружности~$\Omega$ пересекаются в~точке~$X$.
Докажите, что радикальная ось окружностей, вписанных в~криволинейные
треугольники $AXB$ и~$CXD$, проходит через середины дуг $BC$ и~$DA$.

\end{problems}

