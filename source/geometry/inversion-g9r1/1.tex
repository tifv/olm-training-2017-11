% $date: 2017-11-13
% $timetable:
%   g9r1:
%     2017-11-13:
%       3:

\worksheet*{Инверсия}

% $authors:
% - Андрей Юрьевич Кушнир

\begin{problems}

% Первая часть — скорее ликбез по инверсии.

%\item
%Вписанная в~неравнобедренный треугольник $ABC$ окружность с~центром~$I$
%касается его сторон $BC$, $CA$, $AB$ в~точках $A_1$, $B_1$, $C_1$.
%Докажите, что описанные окружности треугольников
%$AIA_{1}$, $BIB_{1}$, $CIC_{1}$ имеют общую точку, отличную от~$I$.
%% Простенькая. Достаточно сообразить где находятся образы вершин треугольника.

\item
Пусть в~сегмент окружности~$\omega$ вписаны две окружности, пересекающиеся
в~точках $A$ и~$B$.
Докажите, что прямая~$AB$ проходит через середину противоположной дуги
окружности.

\item
Вписанная в~треугольник $ABC$ окружность с~центром~$I$ касается сторон~$BC$,
$CA$, $AB$ в~точках $A_1$, $B_1$, $C_1$ соответственно.
Пусть $X$~--- точка пересечения $AI$ и~$B_{1}C_{1}$.
Докажите, что центр описанной окружности треугольника $A_{1}XA$ лежит на~$BC$.

\item
Окружности $\omega_1$, $\omega_2$, $\omega_3$, $\omega_4$ касаются друг друга
внешним образом по~циклу.
Окружность~$\omega$ касается их всех внешним образом в~точках
$X_1$, $X_2$, $X_3$, $X_4$.
Докажите, что касательные к~$\omega$ в~точках $X_1$, $X_3$ пересекаются
на~прямой~$X_{2}X_{4}$.

\item
Дана полуокружность~$\omega$ с~диаметром~$BC$.
Окружность~$\beta$ касается $\omega$ внутренним образом и~отрезка~$BC$
в~точке~$D$.
Прямая~$\ell$ перпендикулярна $BC$ и~касается $\beta$.
Пусть она пересекает дугу~$\omega$ в~точке~$A$ и~отрезок~$BC$ в~точке~$P$.
Докажите, что $AD$ делит угол $BAP$ пополам.

%\item
%Треугольник $ABC$ вписан в~окружность с~центром в~$O$.
%Окружность~$\omega_1$ касается стороны~$AC$ в~точке~$A$ и~проходит через $B$.
%Окружность~$\omega_2$ касается стороны~$AB$ в~точке~$A$ и~проходит через $C$.
%Эти окружности пересекаются в~точке~$K$.
%Докажите, что угол $AKO$~--- прямой.

\item
В~остроугольном треугольнике $ABC$ проведена высота~$AH$.
Точки $K$ и~$L$~--- проекции~$H$ на~стороны $AB$ и~$AC$.
Окружность, описанная около треугольника $ABC$, пересекает прямую~$KL$ в~точках
$P$ и~$Q$ и~прямую~$AH$ в~точках $A$ и~$T$.
Докажите, что~$H$ является инцентром треугольника $PQT$.

\item
На~прямой даны два непресекающихся отрезка.
Найдите геометрическое место точек плоскости, из~которых отрезки видны под
равными углами.

\item
Пусть $AA_{1}$, $AA_{2}$~--- высота и~биссектриса треугольника $ABC$.
Окружность~$\omega$ проходит через точки $A_1$, $A_2$ и~касается вписанной
окружности в~точке~$A_3$ внутренним образом.
Аналогично определяются точки $B_3$, $C_3$ .
Докажите, что прямые $AA_{3}$, $BB_{3}$, $CC_{3}$ пересекаются в~одной точке.
% Очень сложно.

\end{problems}

