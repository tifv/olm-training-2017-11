% $date: 2017-11-16
% $timetable:
%   g9r1:
%     2017-11-16:
%       3:

\worksheet*{Инверсия. Добавка}

% $authors:
% - Андрей Юрьевич Кушнир

\begin{problems}

\item
Дан треугольник $ABC$.
Окружность~$\omega_1$ касается стороны~$AC$ в~точке~$A$ и~проходит через~$B$.
Окружность~$\omega_2$ касается стороны~$AB$ в~точке~$A$ и~проходит через~$C$.
Окружности $\omega_1$ и~$\omega_2$ пересекаются второй раз в~точке~$Q$.
Пусть $O$~--- центр описанной окружности треугольника.
Докажите, что угол $AKO$~--- прямой.

\item
Окружности $S_{1}$ и~$S_{2}$ пересекаются в~точках $A$ и~$B$.
Через точку $B$ проведена прямая $\ell$, вторично пересекающая
окружности $S_{1}$ и~$S_{2}$ в~точках $C$ и~$D$ соответственно.
Точка~$K$ на~окружности~$S_{2}$ такова, что прямые $CA$ и~$AK$ перпендикулярны.
Точка~$L$ на~окружности~$S_{1}$ такова, что прямые $DA$ и~$AL$ перпендикулярны.
Точка~$P$ симметрична точке~$A$ относительно прямой~$\ell$.
Докажите, что точки $A$, $K$, $L$ и~$P$ лежат на~одной окружности.

\item
Ортоцентр~$H$ треугольника $ABC$ лежит на~вписанной в~треугольник окружности.
Докажите, что три окружности с~центрами $A$, $B$, $C$, проходящие через $H$,
имеют общую касательную.

\item
Прямая~$\ell$, параллельная стороне~$BC$ треугольника $ABC$, пересекает отрезки
$AB$ и~$AC$ в~точках $P$ и~$Q$, а~также <<меньшие>> дуги $AB$ и~$AC$ описанной
окружности в~точках $X$ и~$Y$ соответственно.
Окружность с~центром~$I$ вписана в~криволинейный треугольник $BXP$, окружность
с~центром $J$ вписана в~криволинейный треугольник $CYQ$.
Докажите, что $\angle BAI = \angle CAJ$.

\end{problems}

