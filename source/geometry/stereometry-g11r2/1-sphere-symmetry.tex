% $date: 2017-11-07
% $timetable:
%   g11r2:
%     2017-11-07:
%       2:

\worksheet*{Стереометрия-1. Симметрии сфер}

% $authors:
% - Андрей Юрьевич Кушнир

\begin{problems}

\item
Сфера вписана в~многогранный угол.
Докажите, что точки касания сфера со~сторонами угла лежат на~одной окружности.

\item
В~четырехгранный угол вписана сфера.
Докажите, что суммы противоположных плоских углов этого четырехгранного угла
равны.

\item
Докажите, что на~ребрах тетраэдра можно написать по~положительному числу так,
чтобы площадь каждой грани была равна сумме чисел на~ребрах этой грани.

\item
В~закрытой крышкой полусферической вазе лежат четыре одинаковых апельсина
и~грейпфрут.
Апельсины касаются вазы, грейпфрут касается всех апельсинов.
Верно~ли, что точки касания грейпфрута с~апельсинами лежат в~одной плоскости?

\item
В~четырехугольную пирамиду $SABCD$, в~основании которой лежит
параллелограмм $ABCD$, можно вписать сферу.
Докажите, что сумма площадей граней $SAB$ и~$SCD$ равна сумме площадей
граней $SBC$ и~$SDA$.

\item
На~ребрах $SA$, $SB$, $SC$ тетраэдра $SABC$ отмечены точки $A_1$, $B_1$, $C_1$
так, что центр описанной сферы тетраэдра $SA_{1}B_{1}C_{1}$ равноудален
от~точек $A$, $B$, $C$.
Точки $A_2$, $B_2$, $C_2$ симметричны точкам $A_1$, $B_1$, $C_1$ относительно
середин ребер $SA$, $SB$, $SC$ соответственно.
Докажите, что существует сфера, проходящая через
точки $A_2$, $B_2$, $C_2$, $A$, $B$, $C$.

\item
Вписанная сфера тетраэдра $SABC$ касается грани $ABC$ в~точке~$X$, вневписанная
сфера касается грани $ABC$ в~точке~$Y$
(и~продолжений граней $SAB$, $SBC$, $SCA$).
Докажите, что $X$ и~$Y$ изогонально сопряжены относительно треугольника $ABC$
(т.\,е. что $\angle BAX = \angle YAC$ и~т.\,п.).

\item
Три сферы друг друга и~касаются плоскости стола, на~котором они лежат,
в~точках $A$, $B$, $C$.
Четвертая сфера касается их трех и~также касается стола в~точке~$S$.
Докажите, что проекции точки~$S$ на~стороны треугольника $ABC$ служат вершинами
правильного треугольника.

\end{problems}

