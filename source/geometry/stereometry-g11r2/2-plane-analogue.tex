% $date: 2017-11-08
% $timetable:
%   g11r2:
%     2017-11-08:
%       2:

\worksheet*{Стереометрия-2. Плоский аналог задачи}

% $authors:
% - Андрей Юрьевич Кушнир

\begin{problems}

\item
Докажите, что плоскость, проходящая через проекции вершины $A$ тетраэдра $ABCD$
на~биссекторные плоскости двугранных углов при ребрах $BC$, $CD$, $DB$,
параллельна плоскости $BCD$.

\item
Внутри тетраэдра отметили произвольную точку~$X$.
Через каждую вершину тетраэдра провели прямую, параллельную отрезку,
соединяющему $X$ с~точкой пересечения медиан противоположной грани.
Докажите, что четыре такие прямые пересекаются в~одной точке.

\item
В~тетраэдре $ABCD$ проведено сечение плоскостью, перпендикулярной радиусу
описанной сферы, идущему в~вершину~$A$.
Это сечение пересекает ребра $AB$, $AC$, $AD$ в~точках $B_1$, $C_1$, $D_1$.
Докажите, что $B$, $C$, $D$, $B_1$, $C_1$, $D_1$ лежат на~одной сфере.

\item
Вписанная и~вневписанная сферы треугольной пирамиды $ABCD$ касаются ее грани
$BCD$ в~различных точках $X$ и~$Y$.
Докажите, что треугольник $AXY$~--- тупоугольный.

\item
Высоты $AA_{1}$, $BB_{1}$, $CC_{1}$, $DD_{1}$ тетраэдра пересекаются в~одной
точке~$H$.
Докажите, что $H$, $A_1$, и~три точки, делящие
отрезки $BB_{1}$, $CC_{1}$, $DD_{1}$ в~отношении $2 : 1$, считая от~вершин,
лежат на~одной сфере.

\item
Сфера с~центром в~плоскости основания $ABC$ тетраэдра $SABC$ проходит через
вершины $A$, $B$ и~$C$ и~вторично пересекает ребра $SA$, $SB$ и~$SC$
в~точках $A_1$, $B_1$ и~$C_1$ соответственно.
Плоскости, касающиеся сферы в~точках $A_1$, $B_1$ и~$C_1$, пересекаются
в~точке~$O$.
Докажите, что $O$~--- центр сферы, описанной около тетраэдра
\(SA_{1}B_{1}C_{1}\).

%\item
%Вписанная в~тетраэдр $ABCD$ сфера касается его
%граней $BCD$, $CDA$, $DAB$ и~$ABC$ точках $A_1$, $B_1$, $C_1$, $D_1$
%соответственно.
%Рассмотрим плоскость, равноудаленную от~вершины~$A$
%и~плоскости $B_{1}C_{1}D_{1}$ и~три другие аналогичные плоскости.
%Докажите, что центр описанной сферы тетраэдра, образованного этими плоскостями,
%совпадает с~центром описанной сферы тетраэдра $ABCD$.
%% Сложновата, возможно. Но очень-очень крутая.

\end{problems}

