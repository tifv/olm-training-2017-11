% $date: 2017-11-09
% $timetable:
%   g11r2:
%     2017-11-09:
%       2:

\worksheet*{Стереометрия-3. Окружности на~сферах}

% $authors:
% - Артемий Алексеевич Соколов

%\begin{problems}

%\item
%Опишите полярное преобразование в~пространстве, и~докажите принцип
%двойственности.

%\end{problems}

%Пусть дана сфера~$\omega$ и~произвольная точка~$P$, не~совпадающая с~ее
%центром.
%Определим окружность~$S_{P} = \omega \cap \pi_P$, где $\pi_P$~--- полярная
%плоскость точки~$P$.

Для каждой точки~$P$, находящейся вне сферы~$\omega$, обозначим через $S_P$
окружность, содержащую все основания касательных из~$P$ к~$\omega$.

\begin{problems}

\item
Докажите, что
\\
\subproblem
если прямая~$AB$ касается сферы, то~$S_A$ и~$S_B$ касаются.
\\
\subproblem
если плоскость $(ABC)$ касается сферы, то~$S_{A}$, $S_{B}$, $S_{C}$
пересекаются в~одной точке.

\end{problems}

\claim{Мысль}
Таким образом, конфигурация точек в~пространстве переносится на~конфигурацию
окружностей на~сфере.
Далее, применив стереографическую проекцию в~какой-нибудь точке сферы, мы
получим плоскую задачу, которая вполне себе может решиться.

\begin{problems}

\item
Около сферы описан пространственный четырехугольник.
Докажите, что четыре точки касания лежат в~одной плоскости.

\item
В~четырехугольную пирамиду $SABCD$ вписана сфера.
На~ребрах $SA$, $SB$, $SC$, $SD$ отмечены точки $A_1$, $B_1$, $C_1$, $D_1$
соответственно.
Докажите, что вторые касательные плоскости к~сфере, проведенные через ребра
$A_{1}B_{1}$, $B_{1}C_{1}$, $C_{1}D_{1}$, $D_{1}A_{1}$, пересекаются в~одной
точке.

\item
К~сфере проведены две скрещивающиеся касательные $\ell_1$ и~$\ell_2$.
На~них выбираются случайные точки $A_1$ и~$A_2$ соответственно таким образом,
что прямая~$A_{1}A_{2}$ касается сферы в~точке~$X$.
Докажите, что все точки~$X$ лежат на~некоторой окружности.

\item
На~ребрах произвольного тетраэдра выбрано по~точке.
Через каждую тройку точек, лежащих на~ребрах с~общей вершиной, проведена
плоскость.
Докажите, что если три из~четырех проведенных плоскостей касаются вписанного
в~тетраэдр шара, то~и~четвертая плоскость также его касается.

\item
Семь из~восьми вершин многогранника, комбинаторно эквивалентного кубу, лежат
на~сфере.
Докажите, что восьмая вершина также лежит на~этой сфере.

\item
Сфера касается всех ребер тетраэдра $ABCD$, кроме, быть может, ребра~$BD$.
Оказалось, что точки касания сферы с~ребрами $AB$, $BC$, $CD$, $DA$ в~указанном
порядке являются вершинами квадрата.
Докажите, что сфера все-таки касается ребра~$BD$.

\end{problems}

