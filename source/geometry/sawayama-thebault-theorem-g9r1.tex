% $date: 2017-11-15
% $timetable:
%   g9r1:
%     2017-11-15:
%       2:

\worksheet*{Окружности, вписанные в~сегменты}

% $authors:
% - Алексей Александрович Пономарев

\subsection*{Предварительные сведения}

\begin{enumerate}

%\item \emph{(воспоминания из~начальной школы)}
%Дана окружность.
%Через точку~$N$ внутри этой окружности проведены две прямые.
%Докажите, что угол между ними равен полусумме дуг, высекаемых прямыми из~окружности.
%А~что можно сказать, если точка~$N$ вне окружности?
%А~про угол между касательной и~хордой?

\item \emph{Лемма Архимеда.}
В~окружности проведена хорда~$AB$.
Другая окружность касается отрезка~$AB$ в~точке~$K$ и~окружности в~точке~$L$.
Докажите, что прямая~$KL$ проходит через середину дуги, дополняющей дугу~$ALB$
до~окружности.

\item \emph{Лемма о~трезубце.}
Середина дуги~$BC$ (не~содержащая точки~$A$) описанной окружности
треугольника $ABC$ равноудалена от~$B$, $C$ и~центра вписанной окружности.

\item \emph{Критерий касания.}
Дан треугольник $ABC$ и~прямая~$CD$
(точки $D$ и~$B$ лежат в~разных полуплоскостях относительно прямой~$AC$).
Докажите, что окружность, описанная около треугольника $ABC$, касается
прямой~$CD$, если и~только если $\angle ABC = \angle ACD$.

%\item
%Пусть $AC$~--- хорда окружности.
%$B_{1}$~--- середина дуги $AC$.
%Прямая, проходящая через $B_{1}$, пересекает $AC$ в~точке~$K$ и~окружность
%в~точке~$N$.
%Докажите, что $B_{1}K \cdot B_{1}N = {B_{1}C}^2$.

%\item
%Пусть $I$~--- центр окружности, вписанной в~треугольник $ABC$.
%$B_{1}$~--- середина дуги $AC$ окружности, описанной около треугольника $ABC$.
%Прямая, проходящая через $B_{1}$, пересекает $AC$ в~точке~$K$ и~описанную
%окружность в~точке~$N$.
%Докажите, что $\angle BIN = \angle IKN$.

\end{enumerate}

\subsection*{Задачи}

\begin{problems}

\item\emph{Лемма о~сегменте / лемма Саваямы.}
Пусть $D$~--- точка на~стороне~$AC$ треугольника $ABC$.
Рассмотрим окружность, касающуюся отрезков $BD$, $DC$ и~окружности, описанной
около треугольника $ABC$.
Пусть $M$ и~$K$~--- точки касания этой окружности с~$BD$ и~$DC$ соответственно.
Докажите, что прямая~$MK$ проходит через точку~$I$~--- центр окружности,
вписанной в~треугольник $ABC$.

\end{problems}

Заметим, что лемма допускает несколько случаев~--- когда внутреннее касание
заменяется внешним, когда окружность касается не~отрезка, а~его продолжения,
когда точка~$D$ лежит вне отрезка~$AC$.
При этом вместо центра вписанной окружности иногда нужно рассматривать
соответствующие центры вневписанных окружностей.

\begin{problems}

\item \emph{Теорема Виктора Тебо (одна из~многих).}
Пусть $ABC$~--- произвольный треугольник.
$D$~--- произвольная точка на~стороне~$AC$.
$I_{1}$~--- центр окружности, касающейся отрезков $AD$, $BD$ и~описанной около
треугольника $ABC$ окружности.
$I_{2}$~--- центр окружности, касающейся отрезков $CD$, $BD$ и~описанной около
треугольника $ABC$ окружности.
Тогда отрезок~$I_{1}I_{2}$ проходит через точку~$I$~--- центр окружности,
вписанной в~треугольник $ABC$, и~при этом отношение
$I_{1}I : II_{2} = \tg^2{\phi/2}$, где $\phi = \angle BDA$.

\end{problems}

Как и~лемма о~сегменте, теорема Тебо также допускает несколько случаев
расположения окружностей.
\\
Окружности из~теоремы Тебо будем называть
\emph{окружностями Тебо} для точки~$X$.

\begin{problems}

\item
Докажите, что окружности Тебо касаются тогда и~только тогда, когда $X$~---
основание биссектрисы треугольника $ABC$.

\item
К~окружностям Тебо проводится общая внешняя касательная, отличная от~$BC$.
Она пересекает отрезок~$AX$ в~точке~$K$.
Докажите, что прямая, параллельная $BC$ и~проходящая через $K$, касается
вписанной окружности треугольника $ABC$.

\item
Докажите, что окружности Тебо равны (и~равны вписанной окружности) тогда
и~только тогда, когда $X$~--- точка касания вневписанной окружности
треугольника $ABC$ со~стороной~$BC$.

\item
Вневписанная окружность треугольника $ABC$, соответствующая вершине~$C$,
касается продолжения стороны~$AC$ в~точке~$P$.
Рассмотрим окружность~$\omega$, касающуюся $AC$ в~точке~$P$ и~прямой,
проходящей через $B$ параллельно $AC$.
Докажите, что $\omega$ касается описанной окружности треугольника $ABC$.

\item
В~окружность вписан четырехугольник $ABCD$.
Докажите, что четыре точки лежат на одной прямой:
центры вписанных окружностей $I_{1}$ и~$I_{2}$ треугольников $ABD$ и~$ACD$,
центры вневписанных окружностей $I_{3}$ и~$I_{4}$ треугольников $ABC$ и~$BCD$
напротив вершин $C$ и~$B$ соответственно.

%один из частных случаев леммы о сегменте
\item
В~угол с~вершиной~$C$ вписана окружность.
Рассмотрим всевозможные треугольники $ABC$ с~вершинами на~сторонах угла,
описанные вокруг данной окружности.
Докажите, что описанные окружности таких треугольников касаются фиксированной
окружности.

\iffalse

\item
В окружности проведена хорда.
В оба образовавшихся сегмента вписаны окружности.
Прямые, проходящие через точки касания этих окружностей с данной и хордой
повторно пересекают данную окружность в точках $A$ и $B$.
Докажите, что AB~--- диаметр данной окружности.

%тренировка к лемме архимеда и лемме о трезубце
\item
На дугах $AB$ и $BC$ окружности, описанной около треугольника $ABC$, выбраны
соответственно точки $K$ и $L$ так, что прямые $KL$ и $AC$ параллельны.
Докажите, что центры вписанных окружностей треугольников $ABK$ и $CBL$
равноудалены от середины дуги $ABC$.
\emph{(V этап Всероссийской, 10 класс, 2006)}

%тренировка к критерию касания прямой и окружности
\item
Биссектрисы углов $A$ и $C$ треугольника $ABC$ пересекают описанную окружность
этого треугольника в точках $A_0$ и $C_0$ соответственно.
Прямая, проходящая через центр вписанной окружности треугольника $ABC$
параллельно стороне $AC$, пересекается с прямой $A_0C_0$ в точке $P$.
Докажите, что прямая $PB$ касается описанной окружности треугольника $ABC$.
\emph{(IV этап Всероссийской, 9 класс, 2006)}

%тренировка к лемме Архимеда, из Прасолова
\item
На~диаметре~$AB$ окружности~$S$ взята точка~$K$ и~из~нее восстановлен
перпендикуляр, пересекающий $S$ в~точке~$L$.
Окружности $S_{A}$ и~$S_{B}$ касаются окружности~$S$, отрезка~$LK$
и~диаметра~$AB$, а~именно, $S_{A}$ касается отрезка~$AK$ в~точке~$A_{1}$,
а~$S_{B}$ касается отрезка~$BK$ в~точке~$B_{1}$.
Докажите, что $\angle A_{1}LB_{1} = 45^{\circ}$.

%гробик на лемму о сегменте
\item
Окружность касается продолжений сторон $CA$ и~$CB$ треугольника $ABC$, а~также
касается стороны~$AB$ этого треугольника в~точке~$P$.
Докажите, что радиус окружности, касающейся отрезков $AP$, $CP$ и~описанной
около этого треугольника окружности равен радиусу вписанной в~этот треугольник
окружности.
%\emph{(Соросовская олимпиада, 1998)}

\item
Треугольник $ABC$ вписан в~окружность~$S$.
Пусть $A_{0}$~--- середина дуги~$BC$ окружности~$S$, не~содержащей $A$;
$C_{0}$~--- середина дуги~$AB$, не~содержащей $C$.
Окружность~$S_{1}$ с~центром~$A_{0}$ касается $BC$,
окружность~$S_{2}$ с~центром~$C_{0}$ касается $AB$.
Докажите, что центр~$I$ вписанной в~треугольник $ABC$ окружности лежит на~одной
из~общих внешних касательных к~окружностям $S_{1}$ и~$S_{2}$.
%\emph{(V этап Всероссийской, 9 класс, 1999)}

\item
В~треугольнике $ABC$ ($AB < BC$) точка $I$~---центр вписанной окружности,
$M$~--- середина стороны~$AC$, $N$~--- середина дуги $ABC$ описанной
окружности.
Докажите, что $\angle{IMA} = \angle{INB}$.
%\emph{(IV этап Всероссийской, 9 класс, 2005)}

% тренировка к лемме Архимеда, теореме Чевы, из Прасолова
\item
Внутри треугольника $ABC$ взята точка~$X$.
Прямая~$AX$ пересекает описанную окружность в~точке~$A_{1}$.
В~сегмент, отсекаемые стороной~$BC$, вписана окружность, касающаяся дуги~$BC$
в~точке~$A_{1}$, а~стороны~$BC$~--- в~точке~$A_{2}$.
Точки $B_{2}$ и~$C_{2}$ определяются аналогично.
Докажите, что прямые $AA_{2}$, $BB_{2}$ и~$CC_{2}$ пересекаются в~одной точке.

%тренировка к критерию касания прямой и окружности
\item
Окружность, вписанная в~угол с~вершиной~$O$, касается его сторон в~точках $A$
и~$B$, $K$~--- произвольная точка на~меньшей из~двух дуг $AB$ этой окружности.
На~прямой~$OB$ взята точка~$L$ такая, что прямые $OA$ и~$KL$ параллельны.
Пусть $M$~--- точка пересечения окружности~$\omega$, описанной около
треугольника $KLB$, с~прямой~$AK$, отличная от~$K$.
Докажите, что прямая~$OM$ касается окружности~$\omega$.
%\emph{(IV этап Всероссийской, 9 класс, 2001)}

%?
\item
В~окружность вписан прямоугольный треугольник $ABC$ с~гипотенузой~$AB$.
Пусть $K$~--- середина дуги~$BC$, не~содержащей точки~$A$, $N$~--- середина
отрезка~$AC$, $M$~--- точка пересечения луча~$KN$ с~окружностью.
В~точках $A$ и~$C$ проведены касательные к~окружности, которые пересекаются
в~точке~$E$.
Докажите, что угол $EMK$~--- прямой.
%\emph{(ММО, 9 класс, 2003)}

\item
Точка~$I$~--- центр вписанной окружности треугольника $ABC$.
Внутри треугольника выбрана точка~$P$ такая, что
\[
    \angle{PBA} + \angle{PCA} = \angle{PBC} + \angle{PCB}
\, . \]
Докажите, что $AP \geq AI$, причем равенство выполняется
тогда и~только тогда, когда $P$ совпадает с~$I$.
%\emph{(IMO, 2006)}

%гробик на~лемму о~сегменте
\item
Пусть треугольник $ABC$ вписан в~окружность~$\omega$.
$A_{0}$, $B_{0}$~--- точки на~сторонах $BC$ и~$CA$ соответственно, такие что
прямая~$A_{0}B_{0}$ параллельна $AB$.
В~сегменты, стягиваемые хордами $BC$ и~$CA$ окружности~$\omega$,
не~содержащие $A$ и~$B$ соответственно, вписаны
окружности $\omega_{A}$, $\omega_{B}$, касающиеся хорд $BC$ и~$CA$
в~точках $A_{0}$, $B_{0}$.
Докажите, что общая касательная к~окружностям $\omega_{A}$, $\omega_{B}$,
<<ближайшая>> к~$AB$, параллельна $AB$.
%\emph{(Из~материалов летней конференции Турнира городов, 1999)}

\fi

\end{problems}

