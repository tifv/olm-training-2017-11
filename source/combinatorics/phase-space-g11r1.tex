% $date: 2017-11-16
% $timetable:
%   g11r1:
%     2017-11-16:
%       1:

\worksheet*{Мера и~фазовое пространство}

% $authors:
% - Андрей Юрьевич Кушнир

\begingroup
    \ifdefined\mathup \def\piconst{\mathup{\muppi}} \fi
    \ifdefined\uppi \providecommand\piconst{\uppi} \fi
    \providecommand\piconst{\pi}

\begin{problems}

\item
На~плоскости расположены многоугольник~$P$ площади $1$ и~$1000$ точек.
Докажите, что многоугольник можно сдвинуть на~вектор, модуль которого
не~превосходит $\sqrt{\frac{1000}{\pi}}$, так, чтобы сдвинутый многоугольник
не~содержал ни~одну из~точек внутри себя.

\item
В~выпуклый многогранник вписана сфера.
Грань многогранника называется \emph{большой,} если образ сферы при проекции
на~эту грань лежит целиком внутри грани.
Какое наибольшее число больших граней может иметь многогранник?

\item
Из~пункта~$A$ в~пункт~$B$ ведут две непересекающиеся дороги.
Известно, что машины (точки), соединенные веревкой длины меньше $2$, смогли
проехать из~$A$ в~$B$, не~разорвав веревки.
Смогут~ли разъехаться круглые возы радиуса~$1$, если они идут на~встречу друг
другу по~разным дорогам?

\item
Три велосипедиста ездят в~одном направлении по~круглому треку длиной
$300$ метров.
Каждый из~них движется со~своей постоянной скоростью, все скорости различны.
Фотограф сможет сделать удачный снимок велосипедистов, если все они окажутся
на~каком-либо участке трека длиной $d$~метров.
При каком наименьшем $d$ фотограф рано или поздно заведомо сможет сделать
удачный снимок?

\item
В~Москве семь высотных зданий.
Приезжий математик хочет найти точку, из~которой они все видны в~заданном
порядке (считая от~МГУ по~часовой стрелке).
Для любого~ли заданного порядка это возможно?

\item
В~пространстве нарисовано $n$~прямых.
Докажите, что можно выбрать из~них не~менее $\frac{7}{24}n$, попарно
не~перпендикулярных друг другу.

\item
На~лугу, имеющем форму квадрата, имеется круглая лунка.
По~лугу прыгает кузнечик.
Перед каждым прыжком он выбирает вершину и~прыгает по~направлению к~ней.
Длина прыжка равна половине расстояния до~этой вершины.
Сможет~ли кузнечик попасть в~лунку?

%\item
%По~шоссе в~одном направлении едут $10$~автомобилей.
%Шоссе проходит через несколько населенных пунктов.
%Каждый из~автомобилей едет с~некоторой постоянной скоростью в~населенных
%пунктах и~с~некоторой другой постоянной скоростью вне населенных пунктов.
%Для разных автомобилей эти скорости могут отличаться.
%Вдоль шоссе расположено $2011$ флажков.
%Известно, что каждый автомобиль проехал мимо каждого флажка, причем около
%флажков обгонов не~происходило.
%Докажите, что мимо каких-то двух флажков автомобили проехали в~одном и~том~же
%порядке.
%% Давал уже, но пусть в комментах сохранится.

\item
\emph{Отрезком} на~сфере назовем любую дугу любой окружности, полученной как
сечение сферы плоскостью, проходящей через центр сферы.
На~сфере радиуса~$1$ нарисована замкнутая несамопересекающаяся ломаная длиной
меньше $2 \piconst$.
Докажите, что существует полусфера, ее содержащая.

\item
В~пироге радиуса~$R$ запекли монетку радиуса $r < R$.
За~какое минимальное число прямолинейных разрезов можно гарантированно задеть
ножом монетку?

\item
Дан выпуклый многоугольник с~рационально кратными $\piconst$ углами.
Докажите, что в~нем существует периодическая бильярдная траектория.

\end{problems}

\endgroup % \piconst

