% $date: 2017-11-08
% $timetable:
%   g10r1:
%     2017-11-08:
%       3:

\worksheet*{Теорема Рамсея}

% $authors:
% - Александр Михайлович Пешнин

\begin{problems}

\item
На~окружности выбрали 9~точек.
В~каждом образовавшемся треугольнике наибольшую сторону покрасили в~синий цвет.
Докажите, что найдется синий четырехугольник с~диагоналями.

\item
Дан полный граф на~$n$ вершинах.
Двое по~очереди красят его ребра.
Первый красит одно ребро красным, второй красит $100$ ребер в~синий цвет, и~так
повторяется, пока все ребра не~будут покрашены.
Может~ли первый при достаточно больших $n$ добиться появления полного подграфа
со~$100$ вершинами и~всеми красными ребрами?

\item
В~Команде любые три человека образуют либо слаженный, либо
неслаженный коллектив.
Докажите, что в~отряде из~1000 человек можно выбрать или четырех человек, любые
три из~которых образуют слаженный коллектив, либо четырех человек, любые три
из~которых образуют неслаженный коллектив.

\item\claim{Теорема Рамсея для двух цветов}
Для любых чисел $p$, $q$, $k$ ($p, q \geq k$) найдется столь большое число
$N = N(p, q; k)$, что если в~множестве из~$N$ и~более элементов все $k$-наборы
раскрасить в~синий и~красный цвета, то~либо найдется $p$-подмножество,
содержащее только синие $k$-наборы, либо найдется $q$-подмножество, содержащее
только красные $k$-наборы.

\item
Обобщите теорему Рамсея на~случай нескольких цветов.

\item
Пусть $f(N)$ означает минимальное $M$, для которого любое множество
из~$M$ точек в~общем положении содержит выпуклый $N$-угольник.
Докажите, что $f(N)$ конечно.

\item
\subproblem\claim{Теорема Шура}
Все натуральные числа покрашены в~несколько цветов.
Тогда можно выбрать три одноцветных числа $x$, $y$, $z$, для которых
$x + y = z$.
\\
\subproblem
Докажите, что для любого $m \in \mathbb{N}$ и~для любого достаточно большого
простого числа~$p$ сравнение $x^{m} + y^{m} \equiv z^{m} \pmod{p}$ имеет
нетривиальные решения.

\item
Докажите, что существует такое натуральное~$n$, что при любой раскраске всех
непустых подмножеств множества $\{ 1, 2, \ldots, n \}$ в~$1000$ цветов найдутся
два непересекающихся подмножества $A, B \subset \{1, 2, \ldots, n\}$ такие, что
подмножества $A$, $B$ и~$A \cup B$ имеют один и~тот~же цвет.

\end{problems}

