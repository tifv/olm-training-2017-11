% $date: 2017-11-09
% $timetable:
%   g10r2:
%     2017-11-09:
%       1:

\worksheet*{Теорема Рамсея возвращается}

% $authors:
% - Александр Михайлович Пешнин

\begin{problems}

%\item
%На~доске написаны несколько натуральных чисел.
%Среди любых трех чисел есть два, совпадающие в~одном из~разрядов.

\item
В~лагерь приехало несколько детей.
Любые двое дружат, враждуют, либо не~знакомы.
Известно, что среди любых четырех школьников найдутся все три вида отношений.
Докажите, что школьников не~больше $13$.

%\item
%Докажите, что в~задаче про ученых из~предыдущего листка можно уменьшить число
%до~$64$.

\item
\subproblem\claim{Теорема Ван дер Вардена (demo-версия)}
Первые $N$~чисел натурального ряда окрашены в~два цвета.
Докажите, что при достаточно большом $N$ найдется одноцветная арифметическая
прогрессия любой наперед заданной длины.
\\
\subproblem
Верно~ли, что если раскрасить в~два цвета весь натуральный ряд, то~всегда можно
найти одноцветную бесконечную арифметическую прогрессию?

\item
\subproblem
Докажите, что из~достаточно длинной последовательности различных целых
чисел можно выбрать убывающую подпоследовательность длины $m$, либо
возрастающую подпоследовательность длины $n$.
\\
\subproblem
Найдите точную оценку неоходимой длины последовательности.
\\
\subproblem
Можно~ли обобщить задачу на~случай произвольных чисел?

\item
Докажите, что существует такое натуральное~$n$, что при любой раскраске всех
непустых подмножеств множества $\{ 1, 2, \ldots, n \}$ в~$1000$ цветов найдутся
два непересекающихся подмножества $A, B \subset \{1, 2, \ldots, n\}$ такие, что
подмножества $A$, $B$ и~$A \cup B$ имеют один и~тот~же цвет.

\item
На~плоскости отмечено $N$~точек общего положения.
Докажите, что если $N$ достаточно велико, то~существует выпуклый многоугольник
с~заданным числом отмеченных вершин.

\end{problems}

