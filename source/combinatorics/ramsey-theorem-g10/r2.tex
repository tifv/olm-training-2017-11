% $date: 2017-11-08
% $timetable:
%   g10r2:
%     2017-11-08:
%       1:

\worksheet*{Теорема Рамсея}

% $authors:
% - Александр Михайлович Пешнин

\begin{problems}

\item
На~плоскости даны $6$~точек, причем никакие $3$ из~них не~лежат на~одной
прямой, и~все попарные расстояния между ними различны.
В~каждом из~$20$ треугольников, образованном некоторыми тремя из~этих шести
точек, наибольшая сторона покрашена в~красный цвет.
Докажите, что найдется красный треугольник.

\end{problems}

\definition
Рассмотрим полный граф, ребра которого раскрашены в~два цвета.
Определим $R(m,n)$ при $m, n \geq 2$ наименьшее число вершин в~таком графе,
при котором заведомо найдется полный подграф первого цвета на~$m$ вершинах,
либо полный подграф второго цвета на~$n$ вершинах.

\begin{problems}

\item\claim{Теорема Рамсея}
Докажите, что:
\\
\subproblem
$R(m, n) \leq R(m, n - 1) + R(m - 1, n)$ при $m,n \geq 3$;
\\
\subproblem
$R(m, n)$ существует для любых натуральных $m$ и~$n$, больших $1$.

\item
Обобщите теорему Рамсея на~случай нескольких цветов.

\item
Каждый из~$66$ ученых переписывается со~всеми остальными.
В~переписке речь идет только об~одной из~четырех тем.
Докажите, что какие-то трое ученых переписываются по~одной и~той~же теме.

\item\claim{Теорема Шура}
Все натуральные числа покрашены в~несколько цветов.
Тогда можно выбрать три одноцветных числа $x$, $y$, $z$, для которых
$x + y = z$.

\item
Докажите, что
\(
    R((m - 1) (n - 1) + 1, (m - 1) (n - 1) + 1) > (R(m, m) - 1) (R(n, n) - 1)
\).

\item
В~Команде любые три человека образуют либо слаженный, либо
неслаженный коллектив.
Докажите, что в~отряде из~1000 человек можно выбрать или четырех человек, любые
три из~которых образуют слаженный коллектив, либо четырех человек, любые три
из~которых образуют неслаженный коллектив.

\item
Ребра полного графа на~бесконечном числе вершин раскрашены в~два цвета.
Докажите, что можно выбрать одноцветный полный подграф на~бесконечном числе
вершин.

\end{problems}

