% $date: 2017-11-09
% $timetable:
%   g10r1:
%     2017-11-09:
%       2:

\worksheet*{Теорема Рамсея возвращается}

% $authors:
% - Александр Михайлович Пешнин

\begin{problems}

\item
В~лагерь приехало несколько детей.
Любые двое дружат, враждуют, либо не~знакомы.
Известно, что среди любых четырех школьников найдутся все три вида отношений.
Докажите, что школьников не~больше
\\
\subproblem $13$;
\quad
\subproblem $11$.

\item
\subproblem\claim{Теорема ван дер Вардена}
Первые $N$~чисел натурального ряда окрашены в~несколько цветов.
Докажите, что при достаточно большом $N$ найдется одноцветная арифметическая
прогрессия любой наперед заданной длины.
\\
\subproblem
Верно~ли, что если раскрасить в~два цвета весь натуральный ряд, то~всегда можно
найти одноцветную бесконечную арифметическую прогрессию?

\item
Докажите, что из~достаточно длинной последовательности различных чисел можно
выбрать убывающую подпоследовательность длины $m$, либо возрастающую
подпоследовательность длины $n$.
Найдите точную оценку длины последовательности.

\end{problems}

