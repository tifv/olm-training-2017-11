% $date: 2017-11-16
% $timetable:
%   g11r2:
%     2017-11-16:
%       1:

\worksheet*{Подсчёты}

% $authors:
% - Олег Павлович Орлов

\begin{problems}

\item
Назовем раскраску шахматной доски в~три цвета \emph{хорошей}, если в~любом
уголке из~$5$ клеток присутствуют все три цвета (уголок из~$5$ клеток
получается вырезанием из~доски $3 \times 3$ таблицы $2 \times 2$).
Докажите, что хороших раскрасок больше чем $6^8$.

\item
Функции $f(x)$ и~$g(x)$ определены на~целых точках, по~модулю не~превосходящих
$1000$.
Пусть $n$~--- количество таких упорядоченных пар $(x, y)$, что $f(x) = g(y)$,
$m$~--- количество упорядоченных пар $(x, y)$ таких, что $f(x) = f(y)$,
$k$~--- количество упорядоченных пар $(x, y)$ таких, что $g(x) = g(y)$.
Докажите, что $2 n \leq m + k$.

\item
На~отрезке отмечено $1000$ точек на~равном расстоянии.
Половина из~них покрашена в~красный цвет, другая половина~--- в~синий.
Вася разбивает точки на~пары <<красная--синяя>> так, чтобы сумма расстояний
в~парах была максимальной.
Докажите, что этот максимум не~зависит от~раскраски точек.

\item
В~графе $2000$ вершин, причем степень каждой является степенью двойки.
Для каждой вершины подсчитали число маршрутов длины не~более $2$, выходящих
из~нее.
Просуммировали полученные выражения по~всем вершинам.
Докажите, что не~могло получиться $100\,000$.

\item
Найдите максимальное количество ориентированных циклов длины~$3$ в~полном
ориентированном графе из~$14$ вершин.

\item
В~каждой из~трех школ учится $n$~школьников.
Каждый школьник знает ровно $n + 1$ школьников из~других школ.
Докажите, что из~каждой школы можно выбрать по~школьнику так, чтобы они были
попарно знакомы.

\end{problems}

