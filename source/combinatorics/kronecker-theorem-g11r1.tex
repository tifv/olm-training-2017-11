% $date: 2017-11-09
% $timetable:
%   g11r1:
%     2017-11-09:
%       1:

\worksheet*{Применения леммы Кронекера}

% $authors:
% - Иван Викторович Митрофанов

\begin{problems}

\item
Докажите, что некоторая степень тройки начинается с~20172017.

\item
\subproblemx{*}
Можно~ли число 1/10 представить в~виде произведения ста положительных
правильных дробей?
\\
\subproblem
Можно~ли число 1/10 представить в~виде произведения ста положительных конечных
десятичных дробей, меньших единицы?

\item
На~прямой конечное число отрезков суммарной длиной $2.41$ покрашено черным,
в~одной из~черных точек сидит кузнечик.
Кузнечик умеет прыгать по~прямой на~$1$ влево или на~$\sqrt 2$ вправо.
Докажите, что даже если он сам будет выбирать тип прыжка, он не~сможет все
время оставаться на~черной части прямой.

\item
Кузнечик живет на~плоскости и~умеет прыгать на~вектора
$(-1, 0)$, $(0,-1)$ и~$(\sqrt{2}, \sqrt 3)$.
Докажите, что он не~сможет все время оставаться внутри многоугольника площади
меньшей, чем $1 + \sqrt 2 + \sqrt 3$ (многоугольник не~обязан быть выпуклым).

\item
На~окружности длины~1 несколько дуг покрасили в~черный цвет так, что нет двух
черных точек на~расстоянии $\sqrt{2}/2$ (расстояние измеряется по~окружности).
Покажите, что суммарная длина черных дуг может быть равна $0{,}499$.
Может~ли она быть равной $0{,}501$?

\item
Числа $1$, $\alpha_{1}$, $\alpha_2$, \ldots, $\alpha_{n}$ рационально
независимы.
Докажите, что на~окружности длиной~1 можно покрасить черным несколько дуг,
чтобы никакое из~расстояний между черными точками не~было равно никакому
из~$\alpha_{i}$ и~чтобы общая длина черных дуг равнялась $0{,}499$.

\item
Подряд записали первые цифры степеней двойки:
\[
    124136125124{\ldots}
\]
Докажите, что различных блоков по~$13$ цифр подряд в~этом ряду ровно $57$.

\item
У~каждого магического кристалла есть заряд~--- вещественное число.
Если соприкоснутся два кристалла разного заряда, то~более слабый кристалл
несколько секунд будет светиться, а~их заряды сравняются и~станут равны
среднему арифметическому исходных двух зарядов.
Если соприкоснутся два кристалла одинакового заряда, ничего не~произойдет.
У~волшебника есть 7 кристаллов различных зарядов, заряд каждого кристалла он
знает в~точности, и~8-й кристалл, заряд которого он забыл, но~помнит, что он
равняется то~ли 2016, то~ли 2017.
Докажите, что волшебник может, совершая попарные касания и~следя
за~результатами, восстановить заряд 8-го кристалла.

\end{problems}

