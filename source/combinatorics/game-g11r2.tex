% $date: 2017-11-07
% $timetable:
%   g11r2:
%     2017-11-07:
%       3:

\worksheet*{Игры}

% $authors:
% - Олег Павлович Орлов

\begin{problems}

\item
На~доске написано число $60$.
За~ход число можно уменьшить на~любой его натуральный делитель.
Игрок, получивший ноль,~--- проигрывает.
Кто выигрывает при правильной игре?

\item
Двое играют.
На~некоторой клетке шахматной доски находится фишка.
Первый игрок переставляет фишку в~какую-то другую клетку.
Затем фишку можно переставлять только на~большее расстояние, чем на~предыдущем
ходе.
Проигрывает тот, кто не~может сделать ход.
Кто выигрывает при правильной игре?

\item
Двое по~очереди пишут в~клетках кубика $2 \times 2 \times 2$
числа $1, 2, \ldots, 24$ (каждое число один раз).
Второй игрок хочет, чтобы суммы чисел в~клетках каждого кольца из~$8$ клеток,
опоясывающего куб, были одинаковыми.
Может~ли первый ему помешать?

\item
На~бесконечном листе клетчатой бумаги двое по~очереди красят стороны клеток:
первый~--- красным цветом, второй~--- синим.
Нельзя красить одну сторону дважды (изначально ничего не~покрашено).
Может~ли первый игрок создать замкнутую ломаную красного цвета?

\item
Имеется $99!$ молекул.
Двое по~очереди за~один ход съедают не~меньше одной, но~не~больше 1\% молекул.
Проигрывает тот, кто не~может сделать ход.
Кто выигрывает при правильной игре?

\item
Два игрока по~очереди выписывают на~доске в~ряд слева направо произвольные
цифры.
Проигрывает игрок, после хода которого одна или несколько цифр, записанных
подряд, образуют число, делящееся на~$11$.
Кто из~игроков победит при правильной игре?

\item
Оля и~Максим оплатили путешествие по~архипелагу из~$2017$ островов, где
некоторые острова связаны двусторонними маршрутами катера.
Они путешествуют, играя.
Сначала Оля выбирает остров, на~который они прилетают.
Затем они путешествуют вместе на~катерах, по~очереди выбирая остров, на~котором
еще не~были (первый раз выбирает Максим).
Кто не~сможет выбрать остров, проиграл.
Докажите, что Оля может выиграть.

\end{problems}

