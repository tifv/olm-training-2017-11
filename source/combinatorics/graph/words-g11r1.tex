% $date: 2017-11-10
% $timetable:
%   g11r1:
%     2017-11-10:
%       3:

\worksheet*{Слова и~графы}

% $authors:
% - Иван Викторович Митрофанов

\begin{problems}

\item
В~десятичном разложении числа~$x$ после запятой ровно сто подслов длины $100$.
Докажите, что $x$ рациональное.

\item
Докажите, что для любого $k$ существует двоичное слово
длины $2^{k} + k - 1$, в~котором все подслова длины~$k$ различные.

\item
Докажите, что пример в~предыдущей задаче можно получить с~помощью следующего
<<жадного>> алгоритма:
сначала пишется слово из~$k$ нулей, а~потом на~каждом шаге справа приписывают,
если возможно, единицу (то~есть если не~появится одинаковых подслов), а~если
не~получается с~единицей~--- то~ноль.

\item
Министерство Цензуры запретило бесконечный счетный набор конечных двоичных
слов, при этом длина первого слова $100$, второго~--- $100^2$, третьего~---
$100^3$ и~так далее.
Докажите, что существует бесконечное двоичное слово без запрещенных подслов.

\item
Одномерный клетчатый автомат можно описать конечным набором информации:
алфавит, размер окна
(которое можно считать симметричным отрезком из~$2 d + 1$ клеток)
и~для каждого способа заполнить окно~--- по~одной букве.
Найдите алгоритм (не~обязательно быстрый), позволяющий понять, является~ли
автомат
\\
\subproblem
инъективным?
\qquad
\subproblem
сюръективным?

\item
Министерство Цензуры запретило конечное число двоичных слов.
$P(n)$~--- число двоичных слов длины~$n$, не~содержащих запрещенных подслов.
Рассмотрим формальный степенной ряд
\[
    g(x) = P(0) + P(1) x + \ldots + P(k) x^k + \ldots
\]
Докажите, что этот ряд задает рациональную функцию (многочлен делить
на~многочлен).

\end{problems}

