% $date: 2017-11-13
% $timetable:
%   g11r2:
%     2017-11-13:
%       2:

\worksheet*{Графы}

% $authors:
% - Олег Павлович Орлов

\begin{problems}

\item
В~полном графе с~$2017$ вершинами каждое ребро покрашено в~один из~четырех
цветов, причем все четыре цвета присутствуют.
Всегда~ли можно выбрать несколько вершин так, чтобы эти вершины соединяли ребра
ровно трех цветов?

\item
В~стране несколько городов, некоторые пары городов соединены беспосадочными
рейсами одной из~$N$ авиакомпаний, причем из~каждого города есть ровно
по~одному рейсу каждой из~авиакомпаний.
Известно, что из~любого города можно долететь до~любого другого (возможно,
с~пересадками).
Из-за финансового кризиса были закрыты $N - 1$ рейсов, но~ни~в~одной
из~авиакомпаний не~закрыли более одного рейса.
Докажите, что по-прежнему из~любого города можно долететь до~любого другого.

\item
В~городе Цветочном $n$~площадей и~$m$ улиц ($m \geq n + 1$).
Каждая улица соединяет две площади и~не~проходит через другие площади.
По~существующей в~городе традиции улица может называться либо синей, либо
красной.
Ежегодно в~городе происходит переименование:
выбирается площадь и~переименовываются все выходящие из~нее улицы.
Докажите, что вначале можно назвать улицы так, что переименованиями нельзя
добиться одинаковых названий у~всех улиц города.

\item
Несколько деревень соединены дорогами, причем длина каждой дороги меньше
$10$ км.
Известно, из~любой деревни до~любой другой можно добраться, проехав меньше
$10$ км.
Одну дорогу закрыли, но~всё еще можно добраться из~любой деревни до~любой
другой.
Докажите, что это можно сделать, проехав меньше $30$ км.
(Дороги могут быть извилистыми, т.\,е. неравенство треугольника не~обязательно
выполнено.)

\item
Даны натуральные взаимно простые числа $p$ и~$q$.
Вася не~знает, сколько человек придет к~нему на~день рождения:
либо $p$, либо $q$.
На~какое минимальное количество кусков (не~обязательно равных) ему нужно
заранее разрезать торт, чтобы он мог раздать торт поровну как на~$p$ человек,
так и~на~$q$?

\item
Петя поставил на~доску $50 \times 50$ несколько фишек, в~каждую клетку
не~больше одной.
Докажите, что Вася может поставить на~свободные поля этой~же доски не~более 99
новых фишек (возможно, ни~одной) так, чтобы по-прежнему в~каждой клетке стояло
не~больше одной фишки, и~в~каждой строке и~каждом столбце этой доски оказалось
четное количество фишек.

\item
Даны $n$ мальчиков и~$2 n - 1$ конфет.
Докажите, что можно дать каждому мальчику по~конфете так, чтобы мальчику,
которому не~нравится его конфета, не~нравились и~конфеты остальных мальчиков
(чтобы не~создавать предпосылок для драки).
% На~лемму Холла.

\end{problems}

