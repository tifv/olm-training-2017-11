% $date: 2017-11-17
% $timetable:
%   g11r2:
%     2017-11-17:
%       1:

\worksheet*{Комбинаторика}

% $authors:
% - Олег Павлович Орлов

\begin{problems}

\item
Игра происходит на~бесконечной плоскости.
Играют двое: один передвигает одну фишку-волка, другой~--- 50 фишек-овец.
После хода волка ходит одна из~овец, затем, после следующего хода волка, опять
какая-нибудь из~овец и~т.\,д.
И~волк, и~овцы передвигаются за~один ход в~любую сторону не~более, чем на~один метр.
Верно~ли, что при любой первоначальной позиции волк поймает хотя~бы одну овцу?

\item
Дан граф.
Любые две смежные вершины не~имеют общих смежных вершин, а~любые две несмежные
вершины имеют ровно две общие смежные вершины.
Докажите, что степени всех вершин равны.

\item
На~предприятии трудятся 50000 человек.
Для каждого из~них сумма количества его непосредственных начальников и~его
непосредственных подчиненных равна 7.
В~понедельник каждый работник предприятия издает приказ и~выдает копию этого
приказа каждому своему непосредственному подчиненному (если такие есть).
Далее, каждый день работник берет все полученные им в~предыдущий день приказы
и~либо раздает их копии всем своим непосредственным подчиненным, либо, если
таковых у~него нет, выполняет приказы сам.
Оказалось, что в~пятницу никакие бумаги по~учреждению не~передаются.
Докажите, что на~предприятии не~менее 97 начальников, над которыми нет
начальников.

\item
На~столе лежит куча из~более, чем $n^2$~камней.
Петя и~Вася по~очереди берут камни из~кучи, первым берет Петя.
За~один ход можно брать любое простое число камней, меньшее $n$, либо любое
кратное $n$ число камней, либо один камень.
Докажите, что Петя может действовать так, чтобы взять последний камень
независимо от~действий Васи.

\item
Дано натуральное число $n > 2$.
Рассмотрим все покраски клеток доски $n \times n$ в~$k$ цветов такие, что
каждая клетка покрашена ровно в~один цвет, и~все $k$~цветов встречаются.
При каком наименьшем $k$ в~любой такой покраске найдутся четыре окрашенных
в~четыре разных цвета клетки, расположенные в~пересечении двух строк и~двух
столбцов?

\item
В~стране $2000$ городов, некоторые пары городов соединены дорогами.
Известно, что через любой город проходит не~более $N$ различных
несамопересекающихся циклических маршрутов нечетной длины.
Докажите, что страну можно разделить на~$2 N + 2$ республик так, чтобы никакие
два города из~одной республики не~были соединены дорогой.

\item
В~100 ящиках лежат яблоки и~апельсины.
Докажите, что можно так выбрать 34 ящиков, что в~них окажется не~менее трети
всех яблок и~не~менее трети всех апельсинов.

\end{problems}

