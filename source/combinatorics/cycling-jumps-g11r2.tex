% $date: 2017-11-08
% $timetable:
%   g11r2:
%     2017-11-08:
%       1:

\worksheet*{Прыжки по~кругу}

% $authors:
% - Олег Павлович Орлов

\begin{problems}

\item
Даны натуральные $n$ и~$k < n - 1$.
Имеются красные и~синие бусинки.
Составляется круговое ожерелье из~$n$ бусинок.
Оно называется \emph{счастливым}, если в~нем нет двух красных бусинок, между
которыми ровно $k - 1$ бусинок.
Какое наибольшее количество красных бусинок может быть в~счастливом ожерелье?

\item
По~кругу расположены $16$~луночек, одна из~которых отмечена.
Петя и~Вася играют в~следующую игру.
В~начале игры Вася кладет шарик в~одну из~луночек.
Далее за~каждый ход Петя называет натуральные число~$k$
(числа $k$ могут отличаться на~разных ходах), а~Вася перемещает шарик
из~луночки, в~которой он находится, на~$k$ луночек по~часовой либо против
часовой стрелки на~свой выбор.
Сможет~ли Петя играть так, чтобы через несколько ходов шарик гарантированно
попал в~отмеченную луночку?

\item
Окружность разделена точками на~$n$ равных дуг (длину одной дуги примем~за~1).
Кузнечик начинает прыгать с~некоторой точки и~делает последовательно
$n - 1$ прыжков: на~1, на~2, \ldots, на~$n - 1$ по~часовой стрелке
(в~этом порядке).
При каких $n$ кузнечик посетит все отмеченные точки?

\item
Даны натуральные $n$ и~$k$.
На~фирме работают $n$~сотрудников, зарплата каждого из~которых выражается
натуральным числом рублей.
Каждый месяц начальник поднимает зарплату на~1 рубль некоторым $k$~сотрудникам.
При каких $n$ и~$k$ может~ли он наверняка сделать все зарплаты равными?

\item
Петя как-то занумеровал вершины правильного 1001-угольника числами
от~1 до~1001.
Вася первым ходом ставит фишку в~какую-то из~вершин.
Каждым последующим ходом он может передвинуть фишку из~вершины~$A$
в~вершину~$B$, если между ними не~больше 9 других вершин и~число в~$B$ больше
числа в~$A$.
Какое наибольшее количество вершин гарантированно сможет посетить Вася, как~бы
Петя ни~нумеровал вершины?

\item
Окружность разделена точками на~$n$ равных дуг (длину одной дуги примем~за~1).
Кузнечик начинает прыгать с~некоторой точки и~делает $n - 1$ прыжков:
на~1, на~2, \ldots, на~$n-1$ по~часовой стрелке в~\emph{некотором} порядке.
Оказалось, что кузнечик посетил все отмеченные точки.
При каких $n$ такое могло произойти?

\item
На~окружности длины $999$ отмечены $999$ точек, делящих ее на~равные дуги.
В~каждой отмеченной точке стоит фишка.
Назовем \emph{расстоянием} между двумя точками длину меньшей дуги между этими
точками.
При каком наибольшем $n$ можно переставить фишки так, чтобы снова в~каждой
отмеченной точке было по~фишке, а~расстояние между любыми двумя фишками,
изначально удаленными не~более, чем на~$n$, увеличилось?

\end{problems}

