% $date: 2017-11-10
% $timetable:
%   g9r1:
%     2017-11-10:
%       1:

\worksheet*{Кооперативные алгоритмы}

% $authors:
% - Иван Викторович Митрофанов

% $build$matter[print]: [[.], [.]]

\begin{problems}

\item
Фокусник вызывает на~сцену из~зала зрителя и~выходит из~комнаты.
Зритель на~правой руке показывает помощнику фокусника три произвольных пальца
(остальные два загнуты).
Помощник выбирает один из~этих пальцев и~просит зрителя его тоже загнуть.
Фокусник возвращается, видит жест из~двух пальцев и~угадывает, какие три пальца
зритель показывал изначально.
Как фокуснику договориться с~помощником, чтобы фокус всегда удавался?

\item
Помощник фокусника просит одного из~зрителей написать на~доске в~ряд $N$~цифр.
Затем помощник фокусника стирает одну из~них и~покидает помещение.
После этого появляется фокусник.
Глядя на~оставшиеся цифры, фокусник безошибочно отгадывает, какая цифра была
стерта.
\par
Фокусник и~его помощник заранее выбирают число~$N$ так, как им удобно;
фокусник видит, на~каком месте стояла стертая цифра.
Любые способы скрытой коммуникации, кроме явно оговоренных в~задаче, запрещены.
\\
\subproblem Покажите, как организовать этот фокус при $N = 10$.
\\
\subproblem Докажите, что при $N = 9$ такой фокус сделать не~получится.

\item
Вася демонстрирует Пете фокус.
На~столе стоят 7 разных стаканов с~компотом.
Вася выходит из~комнаты, Петя подходи к~столу, один стакан выпивает, а~еще
в~один плюет.
После этого к~столу подходит Коля и~выпивает один из~оставшихся стаканов
(в~который не~плюнули, разумеется!)
Вася возвращается и~угадывает, в~какой стакан плюнул Петя.
\\
\subproblem Как Васе договориться с~Колей, чтобы фокус всегда удавался?
\\
\subproblem А~получится~ли организовать такой фокус, если стаканов 8?

\item
\subproblem
Антону и~Боре надевают на~голову по~шляпе.
Каждая из~шляп может быть черной или белой.
Каждый видит чужую шляпу, но~никто не~видит при этом своей шляпы.
Каждый (не~подглядывая, не~общаясь и~не~подавая друг другу никаких сигналов)
должен попытаться угадать цвет своей шляпы.
Для этого по~команде одновременно каждый из~них должен назвать цвет~---
<<черный>> или <<белый>>.
Если хоть один из~них угадал~--- они выиграли.
Перед тем как всё это произойдет, им дали возможность посовещаться.
Как им следует действовать, чтобы в~любой ситуации выиграть?
\\
\subproblem
То~же самое, но~число цветов увеличивается до~четырех, а~к~Антону и~Боре
присоединяются Вика и~Гена.
\\
\subproblem
Докажите, что из~$n$ людей при $n$ цветах можно договориться так, чтобы хотя~бы
один угадал.
\\
\subproblem
А~если цветов больше, чем людей~--- то~нельзя.


\item
Есть шапки $10$~цветов и~$99$ мудрецов.
Все видят всех, одновременно называют по~цвету.
Какое наибольшее число мудрецов смогут гарантированно угадать свои цвета?

\item
Есть $n$~мудрецов, $n$~цветов, но~одного из~мудрецов посадили в~темный угол,
поэтому серая и~черная шапки на~нем выглядят одинаково.
Остальные цвета на~нем отличимы друг от~друга, а~также от~серого и~черного;
на~всех остальных мудрецах хорошо видны все цвета.
Докажите, что мудрецы не~смогут договориться так, чтобы хотя~бы один
гарантированно угадал свой цвет.

\item
Одиннадцати мудрецам завязывают глаза и~надевают каждому на~голову колпак
одного из~1000 цветов.
После этого им глаза развязывают, и~каждый видит все колпаки, кроме своего.
Затем одновременно каждый показывает остальным одну из~двух карточек – белую
или черную.
После этого все должны одновременно назвать цвет своих колпаков.
Удастся~ли это?
Мудрецы могут заранее договориться о~своих действиях
(до~того, как им завязали глаза);
мудрецам известно, каких 1000 цветов могут быть колпаки.

\item
Дракон поймал трех мудрецов и~устраивает им испытание:
на~голову каждому он надевает шапочку одного из~двух цветов (т.\,е. всего
8 равновероятных возможностей), каждый мудрец видит две чужие шапки и~не~видит
свою.
После этого каждый по~команде может назвать цвет
(а~может и~не~называть ничего).
После этого дракон отпускает мудрецов, если были правильные ответы и~не~было
неправильных, а~в~противном случае~--- сжирает.
Как им договориться перед процедурой, чтобы максимизировать вероятность
спасения?

\item
Четыре мудреца стоят по~кругу, у~них шляпы трех цветов, каждый видит шляпы
соседей, но~не~видит шляпу стоящего напротив.
Есть~ли у~них стратегия, позволяющая хотя~бы одному угадать свой цвет?

\item
Предположим, что мудрецы разбиты на~две группы, и~на~голову каждого мудреца
надела шляпа одного из~$10$ цветов.
Каждый мудрец видит цвета шляп мудрецов из~другой группы и~не~видит цвета шляп
из~своей, все одновременно пытаются угадать свой цвет, предварительно обсудив
стратегию.
\\
\subproblem
Докажите, что если в~одной из~групп 8 мудрецов или меньше, то~у~них это
не~получится.
\\
\subproblem
Докажите, что если в~одной из~групп 9 мудрецов, а~в~другой~--- $10^{10^9}$,
то~у~них есть совместная стратегия.

\end{problems}

