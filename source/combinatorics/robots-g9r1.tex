% $date: 2017-11-08
% $timetable:
%   g9r1:
%     2017-11-08:
%       1:

\worksheet*{Роботы с~конечной памятью}

% $authors:
% - Иван Викторович Митрофанов

\begin{problems}

\itemy{-1}
Докажите, что робот с~конечной памятью и~двумя камнями может обойти все клетки
бесконечной полосы.

\itemy{0}
Докажите, что робот с~конечной памятью без камней не~может обойти бесконечную
полосу.

\item
Докажите, что робот с~4 камнями может обойти всю плоскость.

\item
На~плоскости вдоль прямой $y = 0$ проложен бесконечный забор.
Робот не~может его преодолеть, но~понимает, что находится в~клетке рядом
с~забором.
Докажите, что если у~него есть 1~камень, то~он сможет обойти все клетки верхней
полуплоскости.

\item
Бесконечный горизонтальный ряд и~бесконечный вертикальный ряд клеток плоскости
покрашены в~красный цвет, остальные клетки белые.
Робот умеет видеть цвет клетки по~собой.
Докажите, что он может обойти всю плоскость.

\item
Можно~ли раскрасить клетки бесконечной полосы в~несколько цветов так, чтобы
робот с~конечной памятью мог обойти всю полосу?
Робот может видеть цвета клеток под ним, но~не~может их менять.

\item
Докажите, что робот с~конечной памятью и~одним камнем не~может обойти полосу.

\item
Два робота-десантника приземляются в~две клетки бесконечной полосы, у~одного
из~них есть камень.
Как им встретиться друг с~другом?
Время течет для них синхронно, у~них конечная память и~они не~знают, кто из~них
изначально правее.
Камень видят оба робота, и~двигать его могут тоже оба.

\item
Лабиринт -- это клетчатый прямоугольник, некоторые клеток внутри -- стенки,
через которые робот не~может пройти, но~которые робот видит, если находится
по~соседству.
Также робот может делать пометки в~лабиринте, то~есть писать в~клетке,
в~которой находится, какой-то символ.
\par
Придумайте такой алгоритм для робота с~конечной памятью, чтобы робот обходил
все доступные клетки \emph{любого} лабиринта.
Робота могут поместить в~любую клетку любого лабиринта, изначально в~клетках
никаких символов нет.

\item 
Может~ли робот с~конечной памятью обойти плоскость, если у~него есть 3 камня?

\item
Два робота-десантника приземляются в~две клетки плоскости, у~одного из~них есть
\\
\subproblem по~одному камню;
\\
\subproblem по~два камня.
Могут~ли они встретиться друг с~другом?
Время течет для них синхронно, у~них обоих конечная память и~они не~знают
ничего про взаимное расположение.
Оба робота видят камни друг друга.

\item
Может~ли робот с~конечным числом камней обойти пространство?

\item
У~робота конечная память и~он принимает конечное число сигналов.
Роботы стоят в~шеренгу.
Робот может понимать, что является крайним в~шеренге, и~может передавать
сигналы свои соседям.
Каждая секунда разбита на~два такта: за~один такт роботы пересылают сигналы
соседям.
а~за~второй~--- меняют свои состояния в~зависимости от~прошлых состояний
и~полученных сигналов.
Также у~каждого робота есть кнопочка и~лампочка.
Роботов ставят в~шеренгу в~начальном состоянии и~крайнему нажимают на~кнопочку.
Нужно, чтобы через некоторое время у~всех одновременно (!) загорелись лампочки.
Покажите, как нужно запрограммировать роботов!
Общее число их неизвестно и~может быть любым.

\end{problems}

