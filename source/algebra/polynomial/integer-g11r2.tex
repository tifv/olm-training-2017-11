% $date: 2017-11-08
% $timetable:
%   g11r2:
%     2017-11-08:
%       3:

\worksheet*{Многочлены с~целыми коэффициентами}

% $authors:
% - Антон Сергеевич Гусев

\begingroup
    \def\binom#1#2{\mathrm{C}_{#1}^{#2}}%

\begin{problems}

\item
Многочлен $P(x)$ с~целыми коэффициентами принимает значения $\pm 1$ в~трех
целых точках.
Докажите, что у~$P(x)$ нет целых корней.

\item
Про многочлен $P(x)$ с~целыми коэффициентами известно, что
\( P(P( \ldots P(a) \ldots )) = a \) для какого-то целого~$a$
(сделано $n$~итераций).
Докажите, что $P(P(a)) = a$.

\item
Докажите, что если значения многочлена $P(x)$ целые для любого целого $x$,
то~существуют такие целые $c_{0}, c_{1}, \ldots, c_{n}$, что
\[
    P(x)
=
    c_{n} \binom{x}{n} + c_{n-1} \binom{x}{n-1} + \ldots + c_{0} \binom{x}{0}
\, . \]

\item
Пусть $m$~--- некоторое целое число, а~$R(x)$~--- многочлен с~действительными
коэффициентами степени~$n$ и старшим коэффициентом $a_{n}$.
Оказалось, что $R(x) \kratno m$ при всех целых $x$.
Докажите, что ${n!} \cdot a_{n} \kratno m$.

\item
Докажите, что многочлен $x^{2014} - x^{2013} + 2013 x^{1003} + 3$ не~разложим
на~произведение двух многочленов с~целыми коэффициентами.

\item
Пусть дан многочлен с~целыми коэффициентами $P(x)$ степени $n > 1$
и~натуральное число $k$.
Пусть $Q(x) = P(P( \ldots P(x) \ldots ))$ ($k$~итераций).
Докажите, что существует не~более $n$ целых точек $t$ таких, что $Q(t) = t$.

\end{problems}

\endgroup % \def\binom

