% $date: 2017-11-10
% $timetable:
%   g11r2:
%     2017-11-10:
%       3:

\worksheet*{Неприводимость над $\mathbb{Z}$}

% $authors:
% - Антон Сергеевич Гусев

\claim{<<Критерий>> Эйзенштейна}
Пусть для многочлена $P(x)$ с~целыми коэффициентами верно, что у~него всего
коэффициенты кроме старшего члена делятся на~простое число $p$, и~свободный
член не~делится на~$p^2$.
Тогда $P(x)$ неприводим над $\mathbb{Z}$.

\begin{problems}

\item
Дан многочлен $f(x) = x^{n} + 5 x^{n-1} + 3$.
Докажите, что $f(x)$ неприводим над $\mathbb{Z}$.

\item
Докажите, что многочлен $P(x) = x^{n} + 4$ приводим над $\mathbb{Z}$ тогда и~только тогда, когда $n$ делится на~4.

\item
Пусть $p$~--- простое число.
Докажите, что $F(x) = x^{p-1} + x^{p-2} + \ldots + x + 1$ неприводим над $\mathbb{Z}$.

    \def\digits#1{\overline{\mathstrut#1}}
\item
Пусть $\digits{a_{n} a_{n-1} {\ldots} a_1 a_0}$~--- десятичная запись
простого числа, причем $a_{n} > 1$.
Докажите, что многочлен
$a_{n} x^{n} + a_{n-1} x^{n-1} + \ldots + a_{1} x + a_{0}$ неприводим
над $\mathbb{Z}$.

\item
Пусть $p > 2$~--- простое числа и~$P(x) = x^{p} - x + p$.
Докажите, что 
\\
\subproblem
Все корни $P(x)$ по~модулю не~превосходят $p^{\frac{1}{p-1}}$
\\
\subproblem
$P(x)$ неприводим над $\mathbb{Z}$.

\item
Пусть $a_{1}, a_{2}, a_{3}, \ldots, a_{n}$~--- целые числа.
Найдите все приводимые над $\mathbb{Z}$ многочлены вида:
\\
\subproblem $(x - a_{1}) (x - a_{2}) \ldots (x - a_{n}) - 1$;
\\
\subproblem $(x - a_{1})^2 (x - a_{2})^2 \ldots (x - a_{n})^2 + 1$;
\\
\subproblem $(x - a_{1}) (x - a_{2}) \ldots (x - a_{n}) + 1$.

\end{problems}

