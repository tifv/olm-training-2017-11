% $date: 2017-11-13
% $timetable:
%   g11r1:
%     2017-11-13:
%       1:
%       2:

\worksheet*{Тренировочная олимпиада~--- 2}

% $authors:
% - Андрей Юрьевич Кушнир
% - Олег Павлович Орлов

\begin{problems}
    \def\digits#1{\overline{\mathstrut#1}}%

\item
Пусть $\sqrt{3} = \digits{1{,}x_{1}x_{2}{\ldots}x_{k}{\ldots}}$~--- двоичная запись
числа $\sqrt{3}$.
Докажите, что для каждого натурального $n$ хотя~бы одна цифра из~списка
$x_{n}, x_{n+1}, \ldots, x_{2n}$ равна $1$.

\item
Пусть $A_1$, $B_1$, $C_1$~--- середины сторон треугольника $ABC$.
На~отрезках $BA_{1}$, $CA_{1}$ отмечены точки $X$~и~$Y$ соответственно так, что
$BX = CY$.
Описанная окружность~$\omega$ треугольника $AXY$ вторично пересекает стороны
$AB$ и~$AC$ в~точках $P$~и~$Q$ соответственно.
Пусть $K$, $M$, $N$~--- точки пересечения пар лучей $PX$ и~$QY$,
$PX$ и~$B_{1}A_{1}$, $QY$ и~$C_{1}A_{1}$.
Докажите, что описанная окружность треугольника $KMN$ касается~$\omega$.

\item
Вершины выпуклого $2n$-угольника ($n \geq 2$) лежат в~узлах целочисленной
решетки.
Пусть $S_{n}$~--- минимальное возможное значение его площади.
\\
\subproblem
Докажите, что $S_n \geq \cfrac{n(n - 1)}{2}$.
\\
\subproblem
Докажите, что существует положительное число~$\alpha$ такое, что при всех
натуральных~$n$ выполнено $S_n > \alpha \cdot n^3$.

\end{problems}

