% $date: 2017-11-08
% $timetable:
%   g11r1:
%     2017-11-08:
%       1:
%       2:

\worksheet*{Тренировочная олимпиада~--- 1}

% $authors:
% - Андрей Юрьевич Кушнир

\begin{problems}

\item
Докажите, что при любой расстановке $1000$ детей по~кругу найдется целое число
$100 \leq k \leq 300$, для которого существует промежуток из~$2k$ подряд идущих
детей, удовлетворяющий свойству: среди первых $k$ детей этого промежутка
девочек столько~же, сколько среди последних $k$.

\item
Существуют~ли такой непостоянный многочлен $P(x)$ с~целыми коэффициентами
и~такая функция $f \colon \mathbb{N} \to \mathbb{N}$, что при всех натуральных
$n$ ровно $P(n)$ натуральных чисел $k$ удовлетворяют уравнению
$\underbrace{f (f (\ldots f}_{\text{$n$ раз}}(k))) (k) = k$?

\item
Точка~$H$~--- ортоцентр остроугольного треугольника $ABC$.
Окружности $\omega_C$~и~$\omega_B$ описаны около треугольников $AHB$ и~$AHC$.
Окружность~$\Gamma$ проходит через точки $B$~и~$C$, а~также пересекает отрезки
$BH$ и~$CH$ вторично в~точках $B'$~и~$C'$.
Наконец, окружность~$\gamma$ касается $\omega_C$~и~$\omega_B$ внешним образом,
а~также касается внутренним образом дуги~$BC$ окружности~$\Gamma$, лежащей
внутри $ABC$, в~точке~$T$.
Докажите, что $T$~--- середина дуги~$B'C'$ окружности~$\Gamma$.

\end{problems}

