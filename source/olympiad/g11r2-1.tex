
% $date: 2017-11-11
% $timetable:
%   g11r2:
%     2017-11-11:
%       1:
%       2:

\worksheet*{Тренировочная олимпиада}

\begin{problems}

\item
В~ряд выписаны $23$ натуральных числа.
Докажите, что между ними можно так расставить знаки сложения, умножения
и~скобки, чтобы значение полученного выражения делилось на~$2000$.

\item
На~ребре~$AB$ правильного тетраэдра $ABCD$ отмечена случайная точка~$X$.
Докажите, что ортоцентры всевозможных треугольников $CDX$ лежат
на~фиксированной окружности, не~зависящей от~положения точки~$X$.

\item
Положительные числа $a$, $b$, $c$ удовлетворяют соотношению
$a^2 + b^2 + c^2 = 2ab + 2bc + 2ca$.
Докажите неравенство:
\[
    \frac{a + b + c}{3} \geq \sqrt[3]{2 a b c}
\, . \]

\item
Вершины связного графа покрашены в~$100$ цветов правильным образом, причем для
любой вершины~$u$ и~для любого цвета~$\vartheta$, отличного от~цвета
вершины~$u$, существует единственная вершина~$v$ цвета~$\vartheta$, смежная
с~$u$.
В~графе больше $1000$ вершин.
Докажите, что в~графе есть простой цикл длины не~менее $198$.

\end{problems}

