% $date: 2017-11-14
% $timetable:
%   g9r2:
%     2017-11-14:
%       1:
%       2:
%   g9r3:
%     2017-11-14:
%       1:
%       2:

\worksheet*{Тренировочная олимпиада~--- 1}

\begin{problems}

\item
На~доске записаны числа 1, 2, 3, \ldots, 1000.
Двое по~очереди стирают по~одному числу.
Игра заканчивается, когда на~доске остаются два числа.
Если их сумма делится на~три, то~побеждает тот, кто делал первый ход, если
нет~--- то~его партнер.
Кто из~них выиграет при правильной игре?

%\item
%У~Пети 2017 карточек с~цифрами 1 и~2017 карточек с~цифрами 2.
%Он построил из~них какое-то число.
%После этого Вася может менять 2~карточки местами, отдавая Пете рубль.
%Вася хочет получить число, кратное 11.
%Какое наименьшее число рублей ему гарантированно хватит?

\item
Приведённый квадратный трехчлен $Q(x)$ с~целыми коэффициентами обладает
следующим свойством:
для любого простого $p$ найдется такое целое $k$, что $Q(k)$ и~$Q(k+1)$ делятся
на~$p$.
Докажите, что найдется такое целое $m$, что $Q(m) = Q(m + 1) = 0$.

%\item
%Докажите, что для любого простого $p > 2$ существует единственное
%натуральное~$n$ такое, что $n^2 + n p$~--- точный квадрат.

%\item
%Можно~ли разбить клетчатую доску $12 \times 12$ на~трехклеточные уголки так,
%чтобы каждый горизонтальный и~вертикальный ряд клеток доски пересекал одно
%и~то~же количество уголков?

%\item
%Петя выбрал несколько последовательных натуральных чисел и~каждое записал либо
%красным, либо синим карандашом (оба цвета присутствуют).
%Может~ли сумма наименьшего общего кратного всех красных чисел и~наименьшего
%общего кратного всех синих чисел являться степенью двойки?

\item
Можно~ли при каком-то натуральном~$k$ разбить все натуральные числа
от~$1$ до~$k$ на~две группы и~выписать числа в~каждой группе подряд в~некотором
порядке так, чтобы получились два одинаковых числа?

%\item
%За~круглым столом сидят 30 человек~--- рыцари и~лжецы (рыцари всегда говорят
%правду, а~лжецы всегда лгут).
%Известно, что у~каждого из~них ровно один друг, причем у~рыцаря этот друг~---
%лжец, а~у~лжеца этот друг~--- рыцарь (дружба всегда взаимна).
%На~вопрос <<Сидит~ли рядом с~вами ваш друг?>> сидевшие через одного ответили
%<<да>>.
%Сколько из~остальных могли также ответить <<да>>?
%(Перечислите все варианты и~докажите, что других нет.)

%\item
%Учитель записал Пете в~тетрадь четыре различных натуральных числа.
%Для каждой пары этих чисел Петя нашел их наибольший общий делитель.
%У~него получились шесть чисел: $1$, $2$, $3$, $4$, $5$ и~$N$, где $N > 5$.
%Какое наименьшее значение может иметь число~$N$?

\item
В~белой таблице $2016 \times 2016$ некоторые клетки окрасили черным.
Назовем натуральное число~$k$ удачным, если $k \leq 2016$, и~в~каждом
из~клетчатых квадратов со~стороной~$k$, расположенных в~таблице, окрашено ровно
$k$~клеток.
(Например, если все клетки черные, то~удачным является только число $1$.)
Какое наибольшее количество чисел могут быть удачными?

%\item
%Найдите все такие натуральные $n$, что $n^{n} + 1$ и~$(2n)^{2n} + 1$~---
%простые числа.

\item
На~окружности с~диаметром~$AB$ выбрана точка~$C$.
Через середину меньшей дуги~$AC$ и~середину отрезка~$BC$ проведена
прямая~$\ell$, которая вторично пересекает окружность в~точке~$K$.
Касательные к~окружности в~точках $B$ и~$C$ пересекаются в~точке~$L$.
Докажите, что $KL \perp \ell$.

%\item
%В~треугольнике $ABC$ проведена биссекстриса~$BB_{1}$.
%Перпендикуляр из~$B_1$ на~$BC$ пересекает дугу~$BC$ описанной окружности
%треугольника $ABC$ в~точке~$K$.
%Перпендикуляр из~$B$ на~$AK$ пересекает $AC$ в~точке~$L$.
%Докажите, что точки $K$, $L$ и~середина дуги~$AC$ (не~содержащей точку~$B$)
%лежат на~одной прямой.

%\item
%Дан параллелограмм $ABCD$ с~тупым углом~$B$, в~котором $AD > AB$.
%На~диагонали~$AC$ выбраны такие точки $K$ и~$L$, что $\angle ABK = \angle ADL$
%(точки $A$, $K$, $L$, $C$ различны, причем $K$ лежит между $A$ и~$L$).
%Прямая~$BK$ пересекает окружность~$\omega$, описанную около треугольника $ABC$,
%в~точках $B$ и~$E$, а~прямая~$EL$ пересекает $\omega$ в~точках~$E$ и~$F$.
%Докажите, что $BF \parallel AC$.

\end{problems}

