% $date: 2017-11-09
% $timetable:
%   g9r2:
%     2017-11-09:
%       1:
%       2:
%   g9r3:
%     2017-11-09:
%       1:
%       2:

\worksheet*{Тренировочная олимпиада~--- 1}

% $authors:
% - Аскар Флоридович Назмутдинов
% - Владислав Викторович Новиков
% - Леонид Андреевич Попов
% - Алексей Сергеевич Воропаев

\begin{problems}

\item
Ира задумала $2$ натуральных числа, $a$ и~$b$.
После этого она выписала на~доску числа $a + b$, $a - b$, $a b$ и~$\frac{a}{b}$.
Ира утверждает, что сумма этих чисел равна 1001.
Докажите, что она ошиблась.

\item
Тюрьма имеет форму правильного треугольника со~стороной~$5$.
Она разделена на~$25$ камер имеющих форму правильного треугольника
со~стороной~$1$.
В~камерах живут школьники, сдававшие лажу.
Можно~ли переселить этих школьников в~тюрьму, имеющую форму квадрата
со~стороной~$5$, разбитого на~$25$ квадратных камер со~стороной~$1$, так, чтобы
школьники живущие по~соседству (по~стороне) в~треугольной тюрьме, остались
соседями (по~стороне) в~квадратной.

\item
Центр окружности~$\omega_{1}$ лежит на~окружности~$\omega_{2}$, окружности
пересекаются в~точках $M$, $N$.
На~окружности~$\omega_{1}$ отмечены диаметрально противположные
точки $A$ и~$B$.
Прямая~$AM$ пересекает $\omega_{2}$ в~точке~$C$, а~прямая~$BN$ пересекает
$\omega_{2}$ в~точке~$D$.
Докажите, что $2 DC = AB$.

\item
В~ряд стоят 5~лестниц, на~вершине каждой из~которых лежит банан.
Между некоторыми лестницами натянуты веревки (не~обязательно на~одном уровне)
так, что из~каждой точки исходит не~более одной веревки.
По~каждой лестнице начинает карабкаться вверх обезьяна, при этом каждый раз,
когда она достигает какой-то веревки, она лезет по~ней (по~одной веревке может
лезть несколько обезьян), а~потом карабкается вверх по~лестнице.
Докажите, что независимо от~количества и~расположения веревок каждой обезьяне
достанется банан.

\item
Саша написал на~доске ненулевую цифру и~приписывает к~ней справа по~одной
ненулевой цифре, пока не~выпишет миллион цифр.
Докажите, что на~доске не~более 100 раз был написан точный квадрат.

\end{problems}

